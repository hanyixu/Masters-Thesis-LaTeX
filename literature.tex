%\newpage
\section{Literature Review}

\subsection{Emotions Relate to Online Multiplayer Games}

Players in multiplayer video games experience a wide range of emotions. Players experience excitement and joy during victories, and frustration and anger when defeated. These emotional responses are fundamental to the gaming experience \parencite{cuerdo_exploring_2024}. Research shows that video games can evoke emotions ranging from exhilaration and pride to empathy and even sadness or moral reflection, which suggests that emotion is at the heart of how players interpret and enjoy games \parencite{cuerdo_exploring_2024}. More importantly, the social context of multiplayer play can modulate these emotional outcomes. For example, one study found that playing a violent game in multiplayer (whether cooperatively or competitively with others) led to less negative affect compared to playing the same game alone \parencite{m_knox_all_2015}. This suggests that having a partner or team can buffer the emotional stress of challenging game content, making the experience feel “safer” or less emotionally detrimental for players \parencite{m_knox_all_2015}. Similarly, having a partner or teammate in-game can buffer the emotional stress of challenging content, making the experience feel “safer” or less emotionally detrimental for players \parencite{liu_connecting_2024}. Co-players not only feel happier, but also reported greater companionship and emotional support, which underlines how shared play can enhance emotional benefits and social comfort in game \parencite{liu_connecting_2024}. On the other hand, games that feature intense competition and interactions with opposing online players can also evoke negative emotions during gameplay, potentially leading to toxic behavior. Studies of online team games (e.g., League of Legends) show that toxic emotional outbursts are contagious among players – if one teammate behaves in a hostile or toxic manner, it can spread to others, undermining the group’s experience \parencite{liu_connecting_2024, lee_less_2025}. This contagion effect illustrates that emotions in multiplayer settings don’t just reside within one player; they spread through social interactions, affecting team dynamics and overall enjoyment \parencite{liu_connecting_2024, lee_less_2025}. Because emotions run high in competitive play, researchers have also begun examining how players regulate their feelings – for instance, the gaming community refers to “tilt” as a state of losing control of one’s emotions (often anger or frustration), which is known to impair game performance if not managed \parencite{cregan_playing_2024}. Understanding these emotional experiences and regulating them is crucial, as uncontrolled negative emotions can diminish a player’s performance and enjoyment, whereas positive emotions and camaraderie can significantly enhance motivation, satisfaction, and continued engagement in multiplayer games \parencite{wu_how_2025}. This makes the study of emotional experiences a key pillar for understanding player behavior and well-being in multiplayer contexts.

\subsection{Interactivity-as-Demand Framework (Bowman, 2020)}

\textcite{vorderer_interactivity_2021} proposed the “Interactivity-as-Demand” framework. He categorized interactive demands in video games into four types: cognitive demands, emotional demands, physical demands, and social demands. 
\begin{itemize}
    \item Cognitive demands refer to the attention and mental resources players invest in understanding rules, assessing situations, solving puzzles, and formulating strategies, such as learning a new game's mechanics or quickly identifying key information within complex interfaces.
    \item Emotional demands encompass the emotional investment elicited during game play along with the player's ongoing need to regulate and reassess these feelings.
    \item Physical demands primarily involve operational exertion, such as the difficulty of controlling a controller, keyboard/mouse, or motion-sensing device; the required precision and reaction speed; and whether significant physical movements are needed.
    \item Social demands encompass the social pressures and obligations arising from the presence of real individuals or virtual characters within the game, such as teammates' expectations, spectators' attention, or emotional bonds formed with NPCs or avatars.
\end{itemize}

Within this framework, video games are understood as a media form characterized by “multiple overlapping demands.” Players must simultaneously comprehend rules and monitor the battlefield (cognitive), execute precise maneuvers and perform high-intensity button presses or aiming (physical), process emotional fluctuations from failure, frustration, and victory (emotional), and deliver socially “appropriate” responses to teammates, opponents, or spectators (social). Given the finite nature of human attention and self-control resources, high-intensity demands across various dimensions compete with one another. For instance, in action-heavy shooter games, novice players often allocate most of their attention to “staying alive” and “coordinating with teammates,” leaving little room to monitor chat windows or audience reactions. This explains why studies have found that even when live streaming or spectator conditions are introduced, players' subjective perception of social demands may not significantly increase.

Existing research has begun applying this framework to understand the relationship between stress, challenge, multiplayer dynamics, and workload. First, at the “stress and challenge” level, Bowman integrates the four categories of demands with flow theory. He posits that when task difficulty (demand) aligns with player skill, cognitive and physical demands can transform into a pleasurable “sense of effort.” Players experience a sense of challenge and mastery through sustained engagement without feeling overwhelmed. However, when demands far exceed skill, players slide from “challenging tension” into “overwhelming pressure,” manifesting as persistent tension, frustration, and mental exhaustion. Second, regarding “multiplayer interaction and social dynamics,” research indicates gaming occurs not only on screens but also within social spheres: offline “side-by-side” experiences, online guilds, and streaming platforms mean players pursue not just victory but also fulfill social needs like maintaining relationships, expressing themselves, and considering others' feelings. These social demands can enhance meaning and belonging while potentially intensifying performance pressure and interpersonal conflicts. Third, regarding “Workload, Performance, and Emotional Exhaustion,” Bowman proposes viewing gaming as a multidimensional workload. The interplay of these four demands can either enhance performance and engagement through moderate challenge or, under prolonged high stress, lead to attention depletion, declining performance, and complex negative emotions like fatigue, guilt, and self-blame. This framework provides a pathway for subsequent research to measure the “demand-load-outcome” relationship. For instance, subjective questionnaires like the Video Game Demand Scale can simultaneously assess players' perceived loads across multiple dimensions, linking these to performance metrics and emotional outcomes.

Studying the psychological states of FPS players aligns perfectly with the “Interactivity as Demand” framework. FPS games impose extreme demands on both cognitive and physical capacities. Players must instantly process vast amounts of spatial and tactical information (enemy positions, map structures, economic status, etc.) while executing high-precision aiming and maneuvering. Emotional demands surge rapidly during ranked matches or high-stakes gameplay. Players must regulate their anger and frustration during losses, including scapegoating or poor teammate performance, while managing excitement and impulsivity during winning streaks or “godlike” moments to maintain consistent play. This pressure is pervasive in multiplayer environments. Verbal abuse, in-game taunts from opponents, and blame from teammates amplify perceived social threat. Rankings, win rates, and “highlight reels” tightly bind players' reputations and identities to their in-game performance. Collectively, FPS games impose high cognitive demands, intense emotional pressure, and strong social dynamics simultaneously. This makes them an ideal setting for studying “when gaming shifts from relaxation to stress” and “how multidimensional interactive needs jointly influence stress experiences and flow states.” They also provide a realistic, ecologically valid experimental environment for understanding players' perceptions of workload, performance fluctuations, and emotional exhaustion.

\subsection{Flow Theory and Challenge–Skill Balance}

\textcite{cziksentmihalyi_flow_1990} Flow theory offers another core framework for understanding player experiences. It posits that flow arises from two key conditions: (1) a match between high challenge and high skill; and (2) clear goals and immediate feedback. When a player's abilities align with the game's challenges, they enter a state of deep concentration. Players gradually lose track of time. Beyond these conditions, subsequent research has identified multiple psychological dimensions of flow, such as behavioral engagement, stable attention, perceived control, enhanced self-efficacy, and positive emotions. These psychological outcomes collectively form a highly integrated optimal experience. Within this framework, when the game's challenges and the player's skills become imbalanced, negative emotions and a negative attitude toward the game are likely to arise.
\begin{itemize}
    \item When challenges significantly exceed a player's skill level, they experience stress, frustration, and tension. Players become flustered and overwhelmed by the game content, struggling to process information, execute actions, and regulate emotions.
    \item When challenges are too simple, players experience boredom. For instance, facing easily defeated robots throughout an entire match creates a sense of disengagement. Such experiences lack the immersion and sense of purpose required for flow, resembling mechanical operations or routine tasks instead.
\end{itemize}

Flow is widely applied in gaming and esports research. It explains fluctuations in performance, emotional experiences, and long-term engagement in high-intensity environments. In esports, flow depends not just on task complexity but also on information velocity, team coordination, time pressure, and social dynamics. Flow states often fluctuate during a match. The flow model in interactive media suggests that high interactivity enhances immersion and that a mismatch between challenge and skill can trigger negative emotions.

Flow theory provides the core logic for understanding “when gaming shifts from engagement to burden.” Building upon the “interactivity equals demand” framework, this study posits flow as the key mechanism for interpreting the emergence of negative emotions: when a player's skills fail to match the cognitive, emotional, physical, and social demands imposed by the game, the experience transitions from immersion to tension or anxiety. In other words, the breakdown of the “challenge = skill” equilibrium represents the psychological trigger point for demand overflow. Therefore, this study specifically examines which demands (cognitive, emotional, physical, social) most readily create a “challenge > skill” state. In high-demand environments like FPS games, does demand imbalance directly lead to negative emotions such as increased stress and heightened frustration? Do players exhibit tendencies toward detachment or a desire to reduce interaction when demands overload persistently?

By integrating flow theory and multidimensional needs, this study clarifies how demand imbalance leads to negative emotions and how gaming transitions from enjoyment to stress. It also lays the groundwork for future analysis of stress and loneliness pathways.

\subsection{Mood Management Theory \& Media Use for Emotion Regulation}

Mood Management Theory (MMT) \parencite{reinecke_mood_2016} posits that individuals actively select media content capable of improving their emotional state to achieve relaxation, distraction, mood restoration, or psychological stability. When experiencing negative emotions, heightened stress, or attentional depletion, people tend to choose media formats that reduce discomfort, enhance perceived control, and promote emotional equilibrium. Thus, media consumption serves an emotion regulation function. Video games, as a medium characterized by high interactivity, immediate feedback, and immersive qualities, are widely regarded as an effective tool for mood management.

In this study, MMT plays a crucial role. Emotional regulation through gaming is widely recognized in daily life, so players enter games with clear psychological expectations, hoping for stress relief. Flow theory emphasizes “challenge-skill matching,” while MMT stresses “medium capability-user demand matching.” When both types of matching fail, emotional management can backfire. What should be a relaxing game becomes a burden; what was intended to reduce loneliness instead deepens isolation. This expectation-outcome gap further intensifies players' feelings of frustration and loss of control. 

\subsection{Uses \& Gratifications in Gaming }

Uses \& Gratifications (U\&G) \parencite{ruggiero_uses_2000} posits that audiences actively select media based on their psychological needs, social demands, and emotional states. They seek out media to obtain corresponding gratifications. In video games, U\&G motivations include achievement, competition, social connection, stress relief, and escapism. Players gain satisfaction through acquiring equipment, leveling up, earning kills, and completing missions. In competitive play, they demonstrate skill, earn rankings, and gain recognition from teammates and friends. Some enjoy gaming with friends, meeting new people, and building teamwork and camaraderie. Others prefer single-player or casual games as tools for relaxation, recovery, and emotional relief. Some players immerse themselves in fictional worlds to escape real-world pressures, negative emotions, or social expectations, even temporarily. 

However, U\&G also suggests that a gap emerges when the user's media choice fails to match their initial motivation, resulting in an “expectation–experience gap” that triggers more intense negative emotions\parencite{ruggiero_uses_2000}. Players who intended to socialize through gaming may instead experience loneliness due to teammates' indifference or verbal abuse. Players who want to de-stress may instead become more tense after losing a ranked match. This gap not only diminishes gaming satisfaction but can also heighten stress, frustration, self-blame, or exhaustion. Such discrepancies are common in high-intensity FPS environments. Competitive pressure, volatile social dynamics, and demanding gameplay often prevent players from achieving their goals of relaxation, connection, or achievement.

This theory provides crucial theoretical grounding for this paper. It provides a framework for understanding the mismatch between "anticipated fulfillment" and "actual emotional outcomes." Players typically enter games with specific psychological needs, such as relaxation, connection, or a sense of victory. When actual experiences diverge from these motivations, negative emotions intensify. This effect is greater than if they had no expectations at all. Thus, your research can interpret the emergence of "stress and loneliness" as outcomes of unmet motivations, rather than being solely caused by the game itself. Why does the gaming experience shift from satisfying to stressful and lonely when players' interactive needs go unmet? 

\subsection{Computational Approaches to Studying Player Emotion}

In recent years, the field of emotional development has increasingly turned to computational modeling to address core questions that traditional behavioral research struggles to answer, such as how emotional understanding and regulation abilities actually develop. \textcite{stein_computational_2025} note that developmental psychology research has long focused on descriptive findings, such as age-related differences in emotion recognition or reasoning. However, these findings alone fail to reveal underlying psychological mechanisms. Computational modeling offers a crucial breakthrough: researchers can utilize model structures, such as connectionism and Bayesian inference, to clarify theoretical hypotheses, simulate various developmental mechanisms, and test which hypotheses best explain observed developmental changes through data fitting and model comparison. In other words, the core value of modeling lies not in prediction itself, but in making implicit cognitive processes explicit, providing a mechanistic answer to the question of “why emotional understanding develops in this way.”

Concurrently, in game-based learning research, \textcite{westera_how_2017} proposed that the process of players learning through games can be viewed as a cognitive-emotional-behavioral system. In his paper, 'How people learn while playing serious games: A computational modeling approach,' he used computational modeling to explore how players learn through feedback, task structure, and interactive design in serious games. Importantly, he emphasizes that gaming is not merely a passive process of information reception; rather, it is a dynamic process involving task challenges, failure, retries, immediate feedback, and intrinsic motivation. Framing this inquiry as mechanism research, he further proposes a developmental trajectory model of 'unaware → attempt → internalize' for individuals within games.

This thesis examines how player experiences shift from relaxed to tense, with cumulative cognitive load and even heightened stress, under multi-layered interactive demands. Bowman's Interactivity-as-Demand framework \parencite{vorderer_interactivity_2021} has already highlighted that gaming experiences are fundamentally dynamic systems of cognitive load, emotional engagement, and social feedback. However, most current research focuses on what players ultimately feel, with limited explanation of how these feelings form through interaction. Drawing on computational modeling from affective development and game learning, this paper also aims to move beyond static, outcome-based measures of affect. It instead views player experience as a constantly updated cognitive-affective system. This system evolves in response to factors such as task difficulty, feedback on failures, team interactions, and changes in personal resources. From this mechanism-oriented perspective, I not only explain why high-demand situations cause player stress. I also reveal how demands accumulate and transform throughout interactions, ultimately leading to qualitative shifts in experience. This offers a clearer theoretical foundation for understanding why games shift from relaxation to tension under specific conditions.

\subsection{Task Demand and Mood Repair in Video Games}

Drawing on established MMT and the development of the emerging gaming industry, \textcite{reinecke_games_2009} explored how players choose games under different emotional states (such as boredom or stress) and how these choices influence the process of emotional repair. This study draws on the "emotion-dependent selection theory" (Zillmann \& Bryant, 1985) and the "emotion management theory" (Zillmann, 2000). They posit that people actively choose specific media to alleviate negative emotions. The authors further proposed the concept of "task demand," which refers to the attention and performance demands placed on gaming. The study found an inverted U-shaped relationship between emotional repair and task load: when task load ranges from low to moderate, emotional repair increases; however, when task load is excessively high, repair decreases. This suggests that greater interactivity is not necessarily better; a moderate level of cognitive and performance engagement is most conducive to emotional recovery.

This study is significant in that it is the first to integrate the processes of "selective engagement" and "emotion repair" within video game research, revealing how the formal characteristic of interactive media, "task load," influences emotion regulation. The authors note that while the emotion regulation of traditional media (such as movies and television) primarily relies on content characteristics, video games, due to their high interactivity and attentional demands, offer greater "intervention potential." Moderate interactivity not only distracts attention and reduces negative thoughts, but also allows players to experience a sense of control and efficacy, thereby achieving restorative psychological benefits. However, excessive interactivity demands may lead to overarousal and cognitive fatigue, transforming gaming from a "restorative" experience into a "depleting" one. This study demonstrates that an imbalance between individual emotion regulation mechanisms and those facilitated by games leads to decreased satisfaction, which in turn intensifies negative emotions. The present research investigates the circumstances under which single-player and multiplayer games disrupt this balance and identifies strategies to mitigate these effects. Understanding when and why games cease to provide restorative benefits is essential. Theoretically, this work extends entertainment and flow theories by analyzing how “interactive demand” may transition from a motivating challenge to a psychological burden. Empirically, it elucidates paradoxical player experiences, such as increased stress following competitive sessions or heightened loneliness during solitary play. Practically, the findings offer guidance for game designers and mental health professionals on optimizing challenge, intensity, and social connectedness within interactive environments.




