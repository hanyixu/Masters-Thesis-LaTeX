%\newpage
\section{Literature Review}

\subsection{Interactivity-as-Demand Framework (Bowman, 2020)}

\textcite{vorderer_interactivity_2021} proposed the Interactivity-as-Demand framework, which conceptualizes video games as systems that place multiple forms of demand on players. These demands include cognitive, emotional, physical, social, and controller-related demands. Cognitive demands refer to the attention and mental resources required to understand rules, assess situations, solve problems, and formulate strategies. Emotional demands involve the affective investment elicited during gameplay and the need to regulate emotional responses such as frustration, excitement, or disappointment. Physical demands primarily involve bodily exertion and fatigue, including sustained posture, repetitive movement, or strain during extended play sessions. Controller demands refer to the precision and responsiveness required to operate input devices such as controllers or keyboard and mouse, including reaction speed and control accuracy. Social demands arise from interactions with other players or characters, including coordination, communication, evaluation by teammates, and social pressure.

Players often turn to video games expecting enjoyment, relaxation, or emotional recovery, an assumption consistent with Uses and Gratifications theory. However, these outcomes depend on how game play is experienced moment to moment. Interactivity-as-Demand theory conceptualizes game play as a set of cognitive, emotional, social, physical, and controller demands placed on the player. When these demands are well aligned with player skill and capacity, Flow theory suggests that play can feel immersive and rewarding. When demands exceed available resources, Mood Management theory predicts that media use may fail to regulate emotion and instead contribute to stress or fatigue. Together, these frameworks suggest that gaming can either support emotional recovery or become emotionally taxing depending on how interactive demands are balanced during play. 

Within this framework, video games are understood as a media form characterized by “multiple overlapping demands.” Players must simultaneously comprehend rules and monitor the battlefield (cognitive), execute precise maneuvers and perform high-intensity button presses or aiming (physical), process emotional fluctuations from failure, frustration, and victory (emotional), and deliver socially “appropriate” responses to teammates, opponents, or spectators (social). Given the finite nature of human attention and self-control resources, high-intensity demands across various dimensions compete with one another. For instance, in action-heavy shooter games, novice players often allocate most of their attention to “staying alive” and “coordinating with teammates,” leaving little room to monitor chat windows or audience reactions. This explains why studies have found that even when live streaming or spectator conditions are introduced, players' subjective perception of social demands may not significantly increase.

Existing research has begun applying this framework to understand the relationship between stress, challenge, multiplayer dynamics, and workload. First, at the “stress and challenge” level, Bowman integrates the four categories of demands with flow theory. He posits that when task difficulty (demand) aligns with player skill, cognitive and physical demands can transform into a pleasurable “sense of effort.” Players experience a sense of challenge and mastery through sustained engagement without feeling overwhelmed. However, when demands far exceed skill, players slide from “challenging tension” into “overwhelming pressure,” manifesting as persistent tension, frustration, and mental exhaustion. Second, regarding “multiplayer interaction and social dynamics,” research indicates gaming occurs not only on screens but also within social spheres: offline “side-by-side” experiences, online guilds, and streaming platforms mean players pursue not just victory but also fulfill social needs like maintaining relationships, expressing themselves, and considering others' feelings. These social demands can enhance meaning and belonging while potentially intensifying performance pressure and interpersonal conflicts. Third, regarding “Workload, Performance, and Emotional Exhaustion,” Bowman proposes viewing gaming as a multidimensional workload. The interplay of these four demands can either enhance performance and engagement through moderate challenge or, under prolonged high stress, lead to attention depletion, declining performance, and complex negative emotions like fatigue, guilt, and self-blame. This framework provides a pathway for subsequent research to measure the “demand-load-outcome” relationship. For instance, subjective questionnaires like the Video Game Demand Scale can simultaneously assess players' perceived loads across multiple dimensions, linking these to performance metrics and emotional outcomes.

Studying the psychological states of FPS players aligns perfectly with the “Interactivity as Demand” framework. FPS games impose extreme demands on both cognitive and physical capacities. Players must instantly process vast amounts of spatial and tactical information (enemy positions, map structures, economic status, etc.) while executing high-precision aiming and maneuvering. Emotional demands surge rapidly during ranked matches or high-stakes gameplay. Players must regulate their anger and frustration during losses, including scapegoating or poor teammate performance, while managing excitement and impulsivity during winning streaks or “godlike” moments to maintain consistent play. This pressure is pervasive in multiplayer environments. Verbal abuse, in-game taunts from opponents, and blame from teammates amplify perceived social threat. Rankings, win rates, and “highlight reels” tightly bind players' reputations and identities to their in-game performance. Collectively, FPS games impose high cognitive demands, intense emotional pressure, and strong social dynamics simultaneously. This makes them an ideal setting for studying “when gaming shifts from relaxation to stress” and “how multidimensional interactive needs jointly influence stress experiences and flow states.” They also provide a realistic, ecologically valid experimental environment for understanding players' perceptions of workload, performance fluctuations, and emotional exhaustion.




\subsection{Flow Theory and Challenge-Skill Balance}

\textcite{cziksentmihalyi_flow_1990} Flow theory offers another core framework for understanding player experiences. It posits that flow arises from two key conditions: (1) a match between high challenge and high skill; and (2) clear goals and immediate feedback. When a player's abilities align with the game's challenges, they enter a state of deep concentration. Players gradually lose track of time. Beyond these conditions, subsequent research has identified multiple psychological dimensions of flow, such as behavioral engagement, stable attention, perceived control, enhanced self-efficacy, and positive emotions. These psychological outcomes collectively form a highly integrated optimal experience. Within this framework, when the game's challenges and the player's skills become imbalanced, negative emotions and a negative attitude toward the game are likely to arise.
\begin{itemize}
    \item When challenges significantly exceed a player's skill level, they experience stress, frustration, and tension. Players become flustered and overwhelmed by the game content, struggling to process information, execute actions, and regulate emotions.
    \item When challenges are too simple, players experience boredom. For instance, facing easily defeated robots throughout an entire match creates a sense of disengagement. Such experiences lack the immersion and sense of purpose required for flow, resembling mechanical operations or routine tasks instead.
\end{itemize}

Flow is widely applied in gaming and esports research. It explains fluctuations in performance, emotional experiences, and long-term engagement in high-intensity environments. In esports, flow depends not just on task complexity but also on information velocity, team coordination, time pressure, and social dynamics. Flow states often fluctuate during a match. The flow model in interactive media suggests that high interactivity enhances immersion and that a mismatch between challenge and skill can trigger negative emotions.

Flow theory provides the core logic for understanding “when gaming shifts from engagement to burden.” Building upon the “interactivity equals demand” framework, this study posits flow as the key mechanism for interpreting the emergence of negative emotions: when a player's skills fail to match the cognitive, emotional, physical, and social demands imposed by the game, the experience transitions from immersion to tension or anxiety. In other words, the breakdown of the “challenge = skill” equilibrium represents the psychological trigger point for demand overflow. Therefore, this study examines how players describe experiences that resemble a “challenge > skill” imbalance across different demand types. In high-demand environments like FPS games, does demand imbalance directly lead to negative emotions such as increased stress and heightened frustration? Do players exhibit tendencies toward detachment or a desire to reduce interaction when demands overload persistently?

Flow theory provides a useful lens for understanding when gaming shifts from engagement to burden. Building on the Interactivity-as-Demand framework, this study treats flow as a reference point for interpreting negative emotional experiences. When players perceive that their skills no longer match the cognitive, emotional, physical, or social demands imposed by the game, the experience may transition from immersion to tension or anxiety. Accordingly, this study examines how players describe experiences that resemble a challenge–skill imbalance across different demand types, particularly in high-demand FPS environments where overload may persist over time.




\subsection{Emotions Relate to Online Multiplayer Games}

Players in multiplayer video games experience a wide range of emotions. Players experience excitement and joy during victories, and frustration and anger when defeated. These emotional responses are fundamental to the gaming experience \parencite{cuerdo_exploring_2024}. Research shows that video games can evoke emotions ranging from exhilaration and pride to empathy and even sadness or moral reflection, which suggests that emotion is at the heart of how players interpret and enjoy games \parencite{cuerdo_exploring_2024}. More importantly, the social context of multiplayer play can modulate these emotional outcomes. For example, one study found that playing a violent game in multiplayer, whether cooperatively or competitively with others, led to less negative affect compared to playing the same game alone \parencite{m_knox_all_2015}. This suggests that having a partner or team can buffer the emotional stress of challenging game content, making the experience feel safer or less emotionally detrimental for players \parencite{m_knox_all_2015}. Co-players not only feel happier, but also report greater companionship and emotional support, underscoring how shared play can enhance emotional benefits and social comfort in games \parencite{liu_connecting_2024}.

At the same time, multiplayer environments can amplify negative emotions, particularly under conditions of intense competition and social evaluation. Games that involve ranked play or persistent team interaction frequently evoke frustration, anger, and blame, which can manifest as toxic behavior. Studies of online team games such as League of Legends show that negative emotional outbursts are socially contagious: hostile or toxic behavior by one player can spread to others, undermining group coordination and overall enjoyment \parencite{liu_connecting_2024, lee_less_2025}. This contagion effect highlights that emotions in multiplayer settings do not remain confined to individual players but circulate through social interaction, shaping team dynamics and collective experience.

Beyond individual emotional reactions, prior research emphasizes that gaming communities also develop shared norms for interpreting and regulating emotion. \textcite{kou_regulating_2013} demonstrate that players in online competitive games collectively manage anti-social behavior through discourse, moral framing, and appeals to fairness and responsibility. Emotional expressions such as anger or frustration are therefore embedded within broader processes of community-level sense-making, in which players negotiate whether emotional responses are justified, excessive, or blameworthy.

This perspective is especially relevant for interpreting player discourse in online forums. Player posts do not merely report internal emotional states; they participate in collective evaluations of stress, fairness, and acceptable emotional conduct. As a result, emotional expressions in multiplayer gaming discourse reflect both personal experience and shared community norms. Understanding these dynamics is crucial, as unmanaged negative emotions can impair performance and enjoyment, while positive emotional climates and camaraderie can enhance motivation, satisfaction, and sustained engagement in multiplayer games \parencite{cregan_playing_2024, wu_how_2025}. This makes the study of emotional experiences a key pillar for understanding player behavior and well-being in multiplayer contexts.


\subsection{Mood Management Theory \& Media Use for Emotion Regulation}

Mood Management Theory (MMT) \parencite{reinecke_mood_2016} posits that individuals actively select media content capable of improving their emotional state to achieve relaxation, distraction, mood restoration, or psychological stability. When experiencing negative emotions, heightened stress, or attentional depletion, people tend to choose media formats that reduce discomfort, enhance perceived control, and promote emotional equilibrium. Thus, media consumption serves an emotion regulation function. Video games, as a medium characterized by high interactivity, immediate feedback, and immersive qualities, are widely regarded as an effective tool for mood management.

In this study, Mood Management Theory provides a framework for understanding players’ expectations entering gameplay. Players often turn to games hoping for stress relief or emotional regulation. Flow theory emphasizes challenge–skill matching, while Mood Management Theory highlights the alignment between media characteristics and user needs. When these forms of matching fail, emotional regulation may backfire. Experiences intended to promote relaxation may instead feel burdensome, while attempts to reduce loneliness may fail under certain conditions, intensifying frustration and perceived loss of control.

While Mood Management Theory explains why individuals turn to media for emotional regulation, it does not assume that recovery always succeeds. \textcite{reinecke_guilty_2014} demonstrate that media use can fail to produce recovery when it is accompanied by guilt, pressure, or perceived obligation. In such cases, activities intended to promote relaxation instead intensify stress, resulting in unsuccessful recovery experiences. Their findings highlight that recovery depends not only on media selection, but also on how demands and expectations are experienced during use.

This perspective directly informs the present study’s focus on when gaming becomes emotionally taxing rather than restorative. Competitive pressure, responsibility toward teammates, or repeated failure may transform gameplay from a voluntary leisure activity into a demanding obligation, undermining its potential for emotional recovery.


\subsection{Uses \& Gratifications in Gaming }

Uses \& Gratifications (U\&G) \parencite{ruggiero_uses_2000} posits that audiences actively select media based on their psychological needs, social demands, and emotional states. They seek out media to obtain corresponding gratifications. In video games, U\&G motivations include achievement, competition, social connection, stress relief, and escapism. Players gain satisfaction through acquiring equipment, leveling up, earning kills, and completing missions. In competitive play, they demonstrate skill, earn rankings, and gain recognition from teammates and friends. Some enjoy gaming with friends, meeting new people, and building teamwork and camaraderie. Others prefer single-player or casual games as tools for relaxation, recovery, and emotional relief. Some players immerse themselves in fictional worlds to escape real-world pressures, negative emotions, or social expectations, even temporarily. 

However, U\&G also suggests that a gap emerges when the user's media choice fails to match their initial motivation, resulting in an “expectation–experience gap” that triggers more intense negative emotions \parencite{ruggiero_uses_2000}. Players who intended to socialize through gaming may instead experience loneliness due to teammates' indifference or verbal abuse. Players who want to de-stress may instead become more tense after losing a ranked match. This gap not only diminishes gaming satisfaction but can also heighten stress, frustration, self-blame, or exhaustion. Such discrepancies are common in high-intensity FPS environments. Competitive pressure, volatile social dynamics, and demanding gameplay often prevent players from achieving their goals of relaxation, connection, or achievement.

This perspective provides a framework for understanding mismatches between anticipated gratification and actual emotional outcomes. Players typically enter games with motivations such as relaxation, connection, or achievement. When gameplay experiences diverge from these expectations, negative emotions intensify. The present study therefore interprets stress and loneliness in gaming as outcomes of unmet interactive needs rather than as effects caused solely by the game itself.

Recent large-scale observational research further suggests that the emotional effects of video game play are highly variable rather than uniform. \textcite{johannes_video_2021} find that video game play is neither inherently harmful nor consistently beneficial for well-being, but instead shows small and heterogeneous associations that depend on context. Aggregate indicators such as play time obscure these differences, limiting their ability to explain why gaming supports well-being for some players while contributing to strain for others.

These findings motivate a focus on experiential mechanisms rather than average outcomes. If gaming’s emotional impact depends on how play is experienced, then examining how players describe demand, frustration, and enjoyment provides insight into the conditions under which gaming supports recovery or becomes emotionally taxing.


\subsection{Task Demand and Mood Repair in Video Games}

Drawing on established MMT and the development of the emerging gaming industry, \textcite{reinecke_games_2009} explored how players choose games under different emotional states (such as boredom or stress) and how these choices influence the process of emotional repair. This study draws on the "emotion-dependent selection theory" (Zillmann \& Bryant, 1985) and the "emotion management theory" (Zillmann, 2000). They posit that people actively choose specific media to alleviate negative emotions. The authors further proposed the concept of "task demand," which refers to the attention and performance demands placed on gaming. The study found an inverted U-shaped relationship between emotional repair and task load: when task load ranges from low to moderate, emotional repair increases; however, when task load is excessively high, repair decreases. This suggests that greater interactivity is not necessarily better; a moderate level of cognitive and performance engagement is most conducive to emotional recovery.

\textcite{reinecke_mood_2016}’s study is significant in that it integrates the processes of "selective engagement" and "emotion repair" within video game research, revealing how the formal characteristic of interactive media, "task load," influences emotion regulation. The authors note that while the emotion regulation of traditional media (such as movies and television) primarily relies on content characteristics, video games, due to their high interactivity and attentional demands, offer greater "intervention potential." Moderate interactivity not only distracts attention and reduces negative thoughts, but also allows players to experience a sense of control and efficacy, thereby achieving restorative psychological benefits. However, excessive interactivity demands may lead to overarousal and cognitive fatigue, transforming gaming from a "restorative" experience into a "depleting" one. This study demonstrates that an imbalance between individual emotion regulation mechanisms and those facilitated by games leads to decreased satisfaction, which in turn intensifies negative emotions. The present research investigates the circumstances under which single-player and multiplayer games disrupt this balance and identifies strategies to mitigate these effects. Understanding when and why games cease to provide restorative benefits is essential. Theoretically, this work extends entertainment and flow theories by analyzing how “interactive demand” may transition from a motivating challenge to a psychological burden. Empirically, it elucidates paradoxical player experiences, such as increased stress following competitive sessions or heightened loneliness during solitary play. Practically, the findings offer guidance for game designers and mental health professionals on optimizing challenge, intensity, and social connectedness within interactive environments.





\subsection{Computational Approaches to Studying Player Emotion}

In recent years, the field of emotional development has increasingly turned to computational modeling to address core questions that traditional behavioral research struggles to answer, such as how emotional understanding and regulation abilities actually develop. \textcite{stein_computational_2025} note that developmental psychology research has long focused on descriptive findings, such as age-related differences in emotion recognition or reasoning. However, these findings alone fail to reveal underlying psychological mechanisms. Computational modeling offers a crucial breakthrough: researchers can utilize model structures, such as connectionism and Bayesian inference, to clarify theoretical hypotheses, simulate various developmental mechanisms, and test which hypotheses best explain observed developmental changes through data fitting and model comparison. In other words, the core value of modeling lies not in prediction itself, but in making implicit cognitive processes explicit, providing a mechanistic answer to the question of “why emotional understanding develops in this way.”

Concurrently, in game-based learning research, \textcite{westera_how_2017} proposed that the process of players learning through games can be viewed as a cognitive-emotional-behavioral system. In his paper, 'How people learn while playing serious games: A computational modeling approach,' he used computational modeling to explore how players learn through feedback, task structure, and interactive design in serious games. Importantly, he emphasizes that gaming is not merely a passive process of information reception; rather, it is a dynamic process involving task challenges, failure, retries, immediate feedback, and intrinsic motivation. Framing this inquiry as mechanism research, he further proposes a developmental trajectory model of 'unaware → attempt → internalize' for individuals within games.

This thesis examines how player experiences shift from relaxed to tense under multi-layered interactive demands. While prior research often focuses on emotional outcomes, this study emphasizes how players describe the formation of these experiences through interaction. Drawing on computational approaches from affective development and game learning, player experience is treated as a dynamically evolving cognitive–affective system shaped by task difficulty, feedback, social interaction, and perceived resource availability. From this perspective, the analysis examines how players articulate high-demand situations associated with stress, rather than claiming causal explanations.


\subsection{Theoretical Predictions}

Building on the theoretical frameworks reviewed above, this study advances a set of theoretically grounded expectations regarding how different interactive demands are discussed in player discourse and how they relate to emotional outcomes. Rather than treating these frameworks as competing explanations, this thesis treats them as complementary lenses that describe different stages of the gaming experience, from player motivation to emotional outcomes.

Uses and Gratifications theory suggests that players actively choose games with the expectation that play will serve emotional or psychological needs, such as relaxation, stress relief, or enjoyment \parencite{sherry_video_2006}. From this perspective, discussions of gaming-related stress are not incidental but reflect moments in which expected gratifications fail to materialize. Mood Management theory further suggests that media experiences can support emotional recovery only when they do not introduce additional cognitive or emotional strain \parencite{reinecke_guilty_2014}. When media demands exceed an individual’s available resources, engagement may shift from recovery to depletion.

Interactivity-as-Demand theory provides a framework for specifying how such strain emerges in interactive media contexts \parencite{vorderer_interactivity_2021}. Games impose multiple, simultaneous demands on players, including cognitive, emotional, social, physical, and controller demands. When these demands are well calibrated to player skill and expectations, Flow theory suggests that engagement may be experienced as absorbing and rewarding \parencite{cziksentmihalyi_flow_1990}. When demands overwhelm available capacity or interfere with one another, emotional outcomes are expected to deteriorate.

Based on this integrated theoretical perspective, several expectations guide the present analysis. Cognitive demands are expected to exhibit mixed emotional associations. When players perceive challenges as fair, meaningful, and aligned with their skill level, cognitive demands should co-occur with positive or ambivalent emotional expressions. In contrast, sudden difficulty spikes, unclear objectives, or perceived unfairness are expected to produce negative emotional responses. This pattern reflects the narrow balance required for flow to emerge.

Controller demands are expected to show predominantly negative emotional associations. Difficulties related to input responsiveness, aiming precision, or control reliability directly obstruct player agency and may undermine all other forms of engagement. Because controller failures cannot be compensated for through strategy or social support, frustration is expected to dominate discourse surrounding these demands.

Social demands are expected to amplify emotional intensity rather than determine valence on their own. Cooperative coordination and supportive communication may coexist with positive emotional expressions, particularly in team-based contexts. However, toxic interactions, harassment, or communication breakdowns are expected to intensify stress and frustration, especially when combined with high cognitive or competitive pressure. As a result, social demands are expected to frequently co-occur with both cognitive and emotional demands.

Physical demands are expected to be discussed primarily in relation to fatigue, strain, or exhaustion, particularly in high-intensity competitive play. While physical exertion may occasionally be framed positively in terms of mastery or improvement, prolonged or excessive physical demands are expected to align more closely with negative emotional expressions.

Emotional demands are expected to function as both outcomes and amplifiers of other demands. Emotional responses such as frustration, anxiety, or excitement are not treated as independent demands but as signals of how players experience the cumulative demand load imposed by gameplay. As such, emotional demands are expected to co-occur frequently with cognitive and social demands in player discourse.

These expectations do not imply causal relationships or population-level prevalence. Instead, they provide a theoretical basis for interpreting patterns in player-generated discourse. If observed sentiment distributions and demand co-occurrence patterns align with these expectations, they offer qualitative support for the applicability of Interactivity-as-Demand and related theories to naturally occurring discussions of gaming-related stress and enjoyment.