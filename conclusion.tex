\section{Conclusion}

This thesis examined a central question: when does gaming support emotional recovery, and when does it become emotionally taxing? Drawing on an analysis of  Reddit posts and comments, the findings provide a clear answer. Gaming supports emotional recovery when interactive demands are calibrated to player capacity and function as intended. Recovery is most likely when cognitive demands feel challenging yet fair, allowing players to experience enjoyment, satisfaction, and a sense of accomplishment rather than confusion or overload. Supportive social interaction, including effective teamwork and constructive communication, further reinforces recovery by distributing effort and reducing emotional strain. Smooth controller responsiveness and manageable physical demands also play a critical role by allowing players to remain immersed in play rather than becoming aware of technical barriers or bodily discomfort.

Gaming becomes emotionally taxing when these conditions break down. Emotional strain is most pronounced when controller demands fail, particularly through issues of input responsiveness, control precision, or interface reliability. These failures consistently generate frustration, even in otherwise engaging game contexts. Gaming also becomes taxing when physical demands accumulate over time, producing fatigue or discomfort that interrupts sustained engagement. Cognitive and social demands further contribute to stress when they exceed player capacity, such as during abrupt difficulty spikes, information overload, or exposure to toxic interaction. In these situations, demands no longer challenge players productively but instead compete for limited emotional and attentional resources.

Taken together, these findings demonstrate that emotional outcomes in gaming are shaped not by play itself, but by how multiple interactive demands are balanced and experienced. Gaming functions as a source of emotional recovery when demands remain manageable and aligned with player capacity. It becomes emotionally taxing when demands accumulate, malfunction, or overwhelm available resources. This demand balance offers a unifying lens for understanding why the same game can function as a source of relief in one context and a source of stress in another.