\section{Analysis}

\subsection{Co-occurrence of Interactive Demands}

This section examines how different interactive demands co-appear within the same player discussions. Rather than assuming that cognitive, emotional, physical, controller, and social demands are experienced independently, this analysis explores whether these demands tend to cluster within individual posts and comments.

The analysis uses binary indicators for each demand category. A co-occurrence matrix was constructed by counting how often two demand types appeared together within the same text. In addition, Pearson correlation coefficients were calculated to assess the strength and direction of association between demand categories across the dataset of 1,457 posts and comments. Rather than treating interactive demands as isolated dimensions, co-occurrence is interpreted here as an indicator of compound demand load at the level of discourse. This concept refers to situations in which players describe having to simultaneously manage multiple forms of demand during game play, such as coordinating with teammates while making rapid strategic decisions under emotional pressure. The results indicate that cognitive and emotional demands most frequently co-occurred. This pattern suggests that cognitively demanding game play is often discussed alongside emotional responses, including frustration, stress, or satisfaction. Social demands also commonly appeared alongside cognitive demands, indicating that multiplayer coordination and communication are frequently framed as intensifying cognitive load rather than distributing it.

In contrast, controller and physical demands exhibited weaker correlations with other demand types. These demands were more often discussed as discrete sources of friction, such as control responsiveness or physical discomfort, rather than as part of a broader experiential bundle. Figure~\ref{fig:demand_cooccurrence} presents both the co-occurrence heatmap and the correlation matrix. Together, these visualizations illustrate that expressions of frustration and enjoyment often emerge from the interaction of multiple demands rather than from a single dominant source. Because co-occurrence counts are influenced by overall demand frequency, these findings should be interpreted as reflecting relative coupling patterns rather than strict causal dependence. Future work may apply normalized association measures to further refine these relationships.
 
\begin{figure}[H]
    \centering
    \includegraphics[width=0.95\linewidth]{img/demand_correlations.png}
    \caption{Correlation of Interactive Demand Dimensions}
    \label{fig:demand_cooccurrence}
\end{figure}

\subsection{Demand Patterns Across Subreddits}

To examine whether interactive demands are discussed differently across gaming communities, this section analyzes demand distributions by subreddit. Each subreddit represents a distinct discursive environment shaped by game genre, competitive norms, and community expectations.

For each subreddit, the proportion of posts referencing each demand type was calculated using the same binary indicators applied in previous analyses. This approach allows for comparison of demand salience across communities while holding the demand detection method constant. Subreddit-level analysis is not intended to represent differences between player populations. Instead, it captures variation in community discourse, including what types of experiences are considered noteworthy, problematic, or worthy of discussion within each space. Competitive subreddits tended to show higher proportions of cognitive and social demands, reflecting the strategic coordination and performance pressure emphasized in these communities’ discussions. In contrast, more general gaming communities exhibited a broader distribution of demand types, including greater attention to controller and physical issues.

The original table of raw subreddit percentages was removed because it did not support meaningful theoretical comparison. Raw values alone obscure the interpretive question of why certain demands become salient within specific communities. Instead, subreddit differences are interpreted as reflecting distinct demand ecologies at the level of discourse. Each community foregrounds particular forms of strain or enjoyment based on shared norms, gameplay goals, and tolerance for frustration. For example, high cognitive demand in competitive subreddits may be normalized or even valued, whereas similar demands may be framed as overwhelming in casual communities. These findings reinforce the importance of context when interpreting player discourse. The same interactive demand may be articulated and evaluated differently depending on the social and cultural environment in which play is discussed.

\subsection{Sentiment Analysis of Interactive Demands}

This section examines how players express enjoyment, frustration, and balance in relation to different interactive demands. Sentiment analysis is used to capture affective patterns in player discourse, with a focus on how emotional expression varies systematically across demand categories rather than on absolute sentiment prevalence.

\subsubsection{Rationale and Multi-Method Strategy}

To improve robustness and interpret-ability, this study adopts a multi-method sentiment analysis strategy that integrates three complementary approaches: a domain-specific keyword method, the VADER lexicon-based model, and the DistilBERT transformer model. Each method captures a different layer of affective expression. The keyword method identifies explicit emotional language common in gaming communities, such as references to frustration, enjoyment, or balance. VADER captures sentiment cues characteristic of social media discourse, including negation, emphasis, and punctuation. DistilBERT provides contextual semantic interpretation, allowing sentiment to be inferred even when affect is implicit or indirect.

Agreement rates across the three methods indicate moderate to strong convergence. Keyword and VADER classifications agree on 58.7\% of texts, Keyword and DistilBERT on 72.9\%, and VADER and DistilBERT on 64.0\%. All three methods assign the same sentiment category to 47.8\% of samples. These agreement patterns suggest that while the methods differ in sensitivity, they capture a shared underlying affective structure in player discourse. Rather than treating disagreement as error, this study interprets variation across methods as informative about different forms of emotional expression.

\begin{figure}
    \centering
    \includegraphics[width=0.95\linewidth]{img/negative_sentiment_comparison_gridlight.png}
    \caption{Negative Sentiment by Demand Type Across Three Methods}
    \label{fig:negative_senti_models}
\end{figure}

Using DistilBERT as the primary sentiment indicator, clear differences emerge across the five interactive demand categories. Social demands show the highest proportion of positive sentiment (33.8\%), followed by emotional demands (33.3\%) and cognitive demands (30.0\%). Physical demands register lower positivity (24.5\%), while controller demands show the lowest positive sentiment at 18.1\%. Negative sentiment exhibits an inverse pattern. Controller demands are associated with the highest negative sentiment (80.6\%), followed by physical demands (74.5\%), cognitive demands (67.7\%), emotional demands (65.5\%), and social demands (63.5\%).

These patterns indicate that controller and physical demands function as primary sources of frustration in player discourse when they fail, whereas cognitive and social demands occupy a more balanced emotional space. Cognitive demands, in particular, are frequently described as difficult but acceptable when perceived as fair and learnable.

\subsubsection{Illustrative Examples from Player Discourse}

Numeric sentiment summaries can obscure how players actually frame demand experiences. To make the patterns concrete, Table~\ref{tab:example_posts_sentiment} presents four short examples drawn from the dataset. Usernames are removed, and excerpts are lightly trimmed for length while preserving wording. DistilBERT scores indicate model confidence, and VADER compound scores provide a rule-based reference point.


These examples illustrate why social demands appear polarized in aggregate. Players may praise teamwork and community in one moment, while describing interpersonal breakdown, quitting, or blame in another. Cognitive demand expressions also show why ``challenging but fair'' language can co-exist with enjoyment, whereas controller issues tend to be framed as disruptive and difficult to tolerate.

{
\singlespacing
\small
\setlength{\tabcolsep}{6pt}
\renewcommand{\arraystretch}{1.15}

\begin{longtable}{>{\raggedright\arraybackslash}p{0.22\linewidth}
                  >{\raggedright\arraybackslash}p{0.52\linewidth}
                  >{\raggedright\arraybackslash}p{0.20\linewidth}}

% --- CAPTION & LABEL ---
\caption{Illustrative Reddit Excerpts with Demand Type and Sentiment Scores}
\label{tab:example_posts_sentiment} \\
\hline

% --- FIRST PAGE HEADER ---
\textbf{Example Type} & \textbf{Excerpt (verbatim)} & \textbf{Sentiment Scores} \\
\hline
\endfirsthead

% --- CONTINUATION HEADER (Appears if table breaks to next page) ---
\multicolumn{3}{c}%
{{\bfseries \tablename\ \thetable{} -- continued from previous page}} \\
\hline
\textbf{Example Type} & \textbf{Excerpt (verbatim)} & \textbf{Sentiment Scores} \\
\hline
\endhead

% --- FOOTER ---
\hline
\endfoot

% --- CONTENT ---

Cognitive demand with positive sentiment &
``I love Ana because she has a high skill ceiling, and it feels rewarding to punish divers who don’t expect you to fight back.'' &
DistilBERT: Positive (0.9998) \newline
VADER: 0.7490 \\
\hline

Cognitive demand with negative sentiment &
``Every game feels like mental overload. I try to think ahead, but the amount of information just burns me out.'' &
DistilBERT: Negative (0.9987) \newline
VADER: -0.6123 \\
\hline

Controller demand with negative sentiment &
``Is lag compensation in BF6 the reason I’m so frustrated? My inputs feel delayed no matter what I do.'' &
DistilBERT: Negative (0.9996) \newline
VADER: -0.6705 \\
\hline

Social demand with positive sentiment &
``Very nice work. That was stellar teamwork, and everyone stayed calm even when things went wrong.'' &
DistilBERT: Positive (0.9999) \newline
VADER: 0.4754 \\
\hline

Social demand with negative outcome &
``My friend rage quit and deleted the game last night. Playing with randoms after that just feels miserable.'' &
DistilBERT: Negative (0.9998) \newline
VADER: -0.0258 \\
\hline

Physical demand with negative sentiment &
``After a few hours my hands start hurting and I can’t focus anymore. That’s usually when I stop having fun.'' &
DistilBERT: Negative (0.9979) \newline
VADER: -0.5316 \\
\hline

Compound demand (cognitive + social) &
``I’m trying to track everything, but my teammates keep yelling callouts nonstop. It becomes stressful instead of exciting.'' &
DistilBERT: Negative (0.9991) \newline
VADER: -0.4482 \\
\hline

Compound demand (cognitive + controller) &
``The mechanics are hard enough already, but when the controls feel inconsistent it completely ruins the experience.'' &
DistilBERT: Negative (0.9994) \newline
VADER: -0.5897 \\

\end{longtable}
}