\section{Analysis}

\subsection{Co-occurrence of Interactive Demands}

This section examines how different interactive demands co-appear within the same player discussions. Rather than assuming that cognitive, emotional, physical, controller, and social demands are experienced independently, this analysis explores whether these demands tend to cluster within individual posts and comments.

The analysis uses binary indicators for each demand category. A co-occurrence matrix was constructed by counting how often two demand types appeared together within the same text. In addition, Pearson correlation coefficients were calculated to assess the strength and direction of association between demand categories across the dataset of 1,457 posts and comments. Rather than treating interactive demands as isolated dimensions, co-occurrence is interpreted here as an indicator of compound demand load at the level of discourse. This concept refers to situations in which players describe having to simultaneously manage multiple forms of demand during game play, such as coordinating with teammates while making rapid strategic decisions under emotional pressure. The results indicate that cognitive and emotional demands most frequently co-occurred. This pattern suggests that cognitively demanding game play is often discussed alongside emotional responses, including frustration, stress, or satisfaction. Social demands also commonly appeared alongside cognitive demands, indicating that multiplayer coordination and communication are frequently framed as intensifying cognitive load rather than distributing it.

In contrast, controller and physical demands exhibited weaker correlations with other demand types. These demands were more often discussed as discrete sources of friction, such as control responsiveness or physical discomfort, rather than as part of a broader experiential bundle. Figure~\ref{fig:demand_cooccurrence} presents both the co-occurrence heatmap and the correlation matrix. Together, these visualizations illustrate that expressions of frustration and enjoyment often emerge from the interaction of multiple demands rather than from a single dominant source. Because co-occurrence counts are influenced by overall demand frequency, these findings should be interpreted as reflecting relative coupling patterns rather than strict causal dependence. Future work may apply normalized association measures to further refine these relationships.
 
\begin{figure}
    \centering
    \includegraphics[width=0.75\linewidth]{img/demand_correlations.png}
    \caption{Correlation of Interactive Demand Dimensions}
    \label{fig:demand_cooccurrence}
\end{figure}

\subsection{Demand Patterns Across Subreddits}

To examine whether interactive demands are discussed differently across gaming communities, this section analyzes demand distributions by subreddit. Each subreddit represents a distinct discursive environment shaped by game genre, competitive norms, and community expectations.

For each subreddit, the proportion of posts referencing each demand type was calculated using the same binary indicators applied in previous analyses. This approach allows for comparison of demand salience across communities while holding the demand detection method constant. Subreddit-level analysis is not intended to represent differences between player populations. Instead, it captures variation in community discourse, including what types of experiences are considered noteworthy, problematic, or worthy of discussion within each space. Competitive subreddits tended to show higher proportions of cognitive and social demands, reflecting the strategic coordination and performance pressure emphasized in these communities’ discussions. In contrast, more general gaming communities exhibited a broader distribution of demand types, including greater attention to controller and physical issues.

The original table of raw subreddit percentages was removed because it did not support meaningful theoretical comparison. Raw values alone obscure the interpretive question of why certain demands become salient within specific communities. Instead, subreddit differences are interpreted as reflecting distinct demand ecologies at the level of discourse. Each community foregrounds particular forms of strain or enjoyment based on shared norms, gameplay goals, and tolerance for frustration. For example, high cognitive demand in competitive subreddits may be normalized or even valued, whereas similar demands may be framed as overwhelming in casual communities. These findings reinforce the importance of context when interpreting player discourse. The same interactive demand may be articulated and evaluated differently depending on the social and cultural environment in which play is discussed.

\subsection{Sentiment Analysis of Interactive Demands}

This section examines how players express enjoyment, frustration, and balance in relation to different interactive demands. Sentiment analysis is used to capture affective patterns in player discourse, with a focus on how emotional expression varies systematically across demand categories rather than on absolute sentiment prevalence.

\subsubsection{Rationale and Multi-Method Strategy}

To improve robustness and interpret-ability, this study adopts a multi-method sentiment analysis strategy that integrates three complementary approaches: a domain-specific keyword method, the VADER lexicon-based model, and the DistilBERT transformer model. Each method captures a different layer of affective expression. The keyword method identifies explicit emotional language common in gaming communities, such as references to frustration, enjoyment, or balance. VADER captures sentiment cues characteristic of social media discourse, including negation, emphasis, and punctuation. DistilBERT provides contextual semantic interpretation, allowing sentiment to be inferred even when affect is implicit or indirect.

Agreement rates across the three methods indicate moderate to strong convergence. Keyword and VADER classifications agree on 58.7\% of texts, Keyword and DistilBERT on 72.9\%, and VADER and DistilBERT on 64.0\%. All three methods assign the same sentiment category to 47.8\% of samples. These agreement patterns suggest that while the methods differ in sensitivity, they capture a shared underlying affective structure in player discourse. Rather than treating disagreement as error, this study interprets variation across methods as informative about different forms of emotional expression.

\begin{figure}
    \centering
    \includegraphics[width=0.75\linewidth]{img/negative_sentiment_comparison_gridlight.png}
    \caption{Negative Sentiment by Demand Type Across Three Methods}
    \label{fig:negative_senti_models}
\end{figure}

Using DistilBERT as the primary sentiment indicator, clear differences emerge across the five interactive demand categories. Social demands show the highest proportion of positive sentiment (33.8\%), followed by emotional demands (33.3\%) and cognitive demands (30.0\%). Physical demands register lower positivity (24.5\%), while controller demands show the lowest positive sentiment at 18.1\%. Negative sentiment exhibits an inverse pattern. Controller demands are associated with the highest negative sentiment (80.6\%), followed by physical demands (74.5\%), cognitive demands (67.7\%), emotional demands (65.5\%), and social demands (63.5\%).

These patterns indicate that controller and physical demands function as primary sources of frustration in player discourse when they fail, whereas cognitive and social demands occupy a more balanced emotional space. Cognitive demands, in particular, are frequently described as difficult but acceptable when perceived as fair and learnable.

\subsubsection{Cross-Method Convergence and Divergence}

To evaluate the robustness of the findings, sentiment estimates for each demand type were compared across three complementary methods: a domain-specific keyword approach, the VADER rule-based lexicon, and the DistilBERT transformer model. Overall, the three methods converge on the same relative patterns while differing in absolute sensitivity.

Controller demands stand out as the most negative category for all three approaches: the keyword method classifies 33.3\% of controller-related texts as negative, VADER raises this estimate to 48.6\%, and DistilBERT detects 80.6\% negative sentiment, indicating a strong, cross-method consensus that controller and input issues are framed as highly frustrating in discourse. Across demand types, DistilBERT consistently assigns higher negative sentiment scores than the other two methods, suggesting that its contextual modeling is more sensitive to subtle or implicit negativity that is not captured by surface-level keywords or rule-based heuristics. In contrast, the keyword method behaves as the most conservative estimator: it relies on an explicit, domain-specific lexicon, which makes it precise for overt expressions (e.g., ``tilted,'' ``burned out'') but more likely to miss implicit or sarcastic complaints. VADER’s estimates typically fall between these two extremes, reflecting its design as a general-purpose, social-media-optimized lexicon that accounts for shifters like negation, punctuation, and intensifiers without fully leveraging deep contextual semantics.

\begin{figure}
    \centering
    \includegraphics[width=0.5\linewidth]{img/method_agreement_matrix.png}
    \caption{VADAR-DistilBERT Agreement Matrix}
    \label{fig:V-D_Matrix}
\end{figure}


Table~\ref{tab:three_method_comparison} summarizes sentiment estimates across the three analytical methods for each demand category and provides a basis for direct comparison of both positive and negative classifications.

\begin{table}[H]
\centering
\small
\setlength{\tabcolsep}{6pt}
\caption{Three-Method Sentiment Comparison}
\label{tab:three_method_comparison}

\begin{tabular}{lrrrrr}
\hline
\textbf{Metric} &
\textbf{Cognitive} &
\textbf{Emotional} &
\textbf{Physical} &
\textbf{Controller} &
\textbf{Social} \\
\hline

Count &
254 & 249 & 94 & 72 & 222 \\

Key + (\%) &
30.71 & 53.41 & 25.53 & 31.94 & 31.08 \\

Key -- (\%) &
20.08 & 40.96 & 24.47 & 33.33 & 16.22 \\

VADER + (\%) &
61.42 & 55.02 & 53.19 & 47.22 & 56.76 \\

VADER -- (\%) &
31.89 & 39.36 & 39.36 & 48.61 & 30.18 \\

DBERT + (\%) &
29.92 & 33.33 & 24.47 & 18.06 & 33.78 \\

DBERT -- (\%) &
68.11 & 65.46 & 74.47 & 80.56 & 63.96 \\
\hline
\end{tabular}
\end{table}

The three-method comparison further clarifies how these differences manifest across positive and negative sentiment patterns. As shown in Table~\ref{tab:three_method_comparison}, VADER consistently reports the highest proportion of positive sentiment across all demand categories, often exceeding 50\% positive for emotional, social, and cognitive demands. DistilBERT yields more moderate positive rates, reflecting a more balanced classification between positive and negative categories when contextual cues are mixed. The keyword method systematically reports the lowest positive sentiment percentages, mirroring its conservative, lexicon-bound nature.

On the negative side, DistilBERT detects substantially more negativity, ranging from about 60\% to over 80\% negative across demands, whereas VADER’s negative estimates cluster in the 30--50\% range, and the keyword method yields the lowest negative percentages (roughly 15--40\%). Importantly, these quantitative differences do not undermine the core findings; instead, they provide complementary perspectives on the same underlying affective landscape. All three methods agree on the ordering of negativity—controller $>$ physical $>$ emotional $>$ cognitive $>$ social—which strongly reinforces the conclusion that interaction demands linked to control and physical effort are most likely to be articulated as stressful, while social and cognitive demands, though sometimes negative, are comparatively less aversive in player discourse.

\subsection{Summary: How Players Express Emotion Through Demands}

Across all sentiment analyses, players express enjoyment and frustration in ways that consistently map onto the five interactive demand types. Enjoyment is strongest with social demands, where players value community, teamwork, and positive interactions. Cognitive demands evoke moderate enjoyment and are often described as stimulating when challenges are well designed. In contrast, controller demands elicit the weakest enjoyment, with positive sentiment emerging primarily when systems function seamlessly rather than as a source of active pleasure. Frustration peaks with controller demands; input delay, inconsistency, or mechanical failures are frequently framed as intensely negative experiences in discourse. Physical demands are also commonly associated with frustration, especially when discussions reference fatigue or discomfort.

Meanwhile, cognitive and emotional demands produce moderate levels of frustration in player discourse. These demands are often described as difficult but manageable, fitting expectations for skill-based play. Players rarely discuss balancing demands and emotions directly, but stable positive sentiment frequently accompanies cognitive demands. Players describe `challenging but fair' encounters as enjoyable, suggesting that well-calibrated cognitive load is articulated as fostering engagement rather than detracting from the experience. Social demands are polarizing, driving both peak positive and negative expressions. Cognitive demands occupy a balanced middle ground when well tuned, combining challenge with perceived fairness in player discussions.
