Video games are often pursued for relaxation and social connection, yet players frequently report experiences of stress, frustration, and emotional exhaustion, particularly in online multiplayer environments. This thesis investigates the paradox of why games stop feeling fun by analyzing how players articulate interactive demands and affective experiences in naturally occurring online discourse. Grounded in Bowman’s Interactivity-as-Demand framework and informed by Flow theory and Mood Management perspectives, the study conceptualizes game play as a system of competing cognitive, emotional, social, physical, and controller-related demands. Using an observational computational approach, 1,457 posts and comments were collected from eleven gaming-related subreddits, focusing on competitive first-person shooter communities. The analysis employs a triangulated strategy combining theory-driven multi-label coding for demand detection with a multi-method sentiment analysis using keyword-based methods, VADER, and a fine-tuned DistilBERT model. Results indicate that cognitive and emotional demands are the most frequently discussed dimensions, often co-occurring to reflect the high-stakes nature of competitive play. While less frequent, controller and physical demands are associated with disproportionately high negative sentiment, functioning as "hygiene factors" that disrupt enjoyment when they fail. Furthermore, social demands exhibit a polarized sentiment profile, acting as both a source of deep satisfaction through teamwork and intense frustration through toxicity. The findings suggest that player frustration is driven less by high demand intensity alone than by the misalignment of these demands with player resources and expectations. This study extends the Interactivity-as-Demand framework by demonstrating the compound nature of gameplay stressors and illustrates the value of computational methods for understanding player experience at scale.
\\
\\
\textit{Keywords: Interactivity-as-Demand, Player Experience, Natural Language Processing, Sentiment Analysis, Multiplayer Online Games }
 