Video games are often understood as sources of enjoyment, relaxation, and social connection. However, players frequently report that online multiplayer play becomes emotionally demanding rather than restorative. This thesis examines when and why games stop feeling fun by analyzing how players articulate interactive demands and affective experiences in naturally occurring online discourse. Grounded in Bowman’s Interactivity-as-Demand framework and informed by flow theory and mood management perspectives, the study conceptualizes gameplay experience as shaped by multiple, coexisting demands, including cognitive, emotional, social, physical, and controller-related demands. Rather than treating these demands as independent dimensions, the thesis emphasizes how their alignment and misalignment shape perceived enjoyment and frustration. Using an observational computational approach, the study analyzes 1,457 posts and comments collected from eleven gaming-related subreddits, with a focus on competitive first-person shooter communities. Interactive demands are identified through theory-driven, multi-label coding, while affective expression is examined using a triangulated sentiment analysis strategy combining keyword-based methods, the VADER lexicon, and a transformer-based DistilBERT model. Results indicate that cognitive, emotional, and social demands are most frequently discussed, whereas controller and physical demands appear less often but are associated with disproportionately negative sentiment. Co-occurrence analysis shows that demands rarely appear in isolation, with cognitive and emotional demands frequently clustering and social demands often intensifying cognitive load. These findings suggest that player frustration is less a function of high demand alone than of compound demand load and poor calibration between effort, control, and perceived fairness. The thesis extends the Interactivity-as-Demand framework by demonstrating how multiple demand dimensions interact in player discourse and illustrates the value of theory-driven computational analysis for studying gameplay experience at scale.
