\section{Plannings: This part do not go to thesis}
Reddit has emerged as a major hub for gaming communities, hosting countless subreddits where players discuss games, share experiences, and react to new developments. Unlike ephemeral in-game chats, Reddit’s asynchronous discussion threads remain publicly archived, allowing researchers to longitudinally trace community sentiments and conversations over time \parencite{bergstrom_reddit_2021} This makes Reddit an invaluable resource for studying how player emotions evolve and how collective discourses form around games. Leveraging perspectives from mass communication and HCI (human-computer interaction), the proposed research will examine sentiment and emotional discourse in Reddit gaming communities.



\subsection{Emotional Expression in Online Gaming Communities}

ePlayers often use online forums to voice their excitement, frustrations, and personal experiences with games, effectively turning platforms like Reddit into emotional echo chambers of the gaming experience. Recent research underscores the value of large-scale sentiment analysis in capturing these player emotions. For example, NLP-based analyses have been applied to vast troves of game-related content (user reviews, chat logs, forum posts) to identify prevailing positive and negative sentiments 


Emotional expressions on Reddit do more than convey individual feelings – they often serve a social function in building community identity. Theoretical work in mass communication and social psychology suggests that shared affect can be a glue for group identity and belonging. Collective identity refers to a shared sense of “we-ness” or group membership that distinguishes insiders from outsiders \parencite{an_multiple_2025}

In online gaming subreddits, collective identity is constructed through common references, memes, values, and often through shared emotional stances towards the game or outside groups (such as rival game communities or game developers). Positive emotions like excitement and joy can foster a sense of camaraderie among subreddit members (e.g. fans uniting in hype over a new release), while negative emotions (anger, frustration) can likewise create solidarity, especially if directed at a common adversary or issue (for example, collective outrage at a controversial update). Emotions become part of the community narrative – stories the group tells about itself, what it loves or hates, and what it stands for. Prior research illustrates this dynamic. \textcite{bergstrom_reddit_2021} observed that on franchise-specific subreddits, players’ nostalgia for older titles kept them loyal to their original community even when new games were released.

\subsection{Sentiment Dynamics Across the Game Lifecycle}

\textcite{baig_reddit_2024} note the importance of tracking evolving sentiments on Reddit when new technologies are introduced in games (e.g. the surge of excitement followed by backlash in discussions of blockchain or NFT features in gaming). Bergstrom \& Poor’s study of subreddit migrations, while focused on user movement, implicitly reflects lifecycle sentiment: even as new games launched, the older game communities retained users through positive emotional factors like nostalgia \parencite{bergstrom_reddit_2021}, indicating that initial hype for the new title did not fully override the enduring affection for the predecessor.

\subsection{Tech Part}

We will gather data from a selection of relevant gaming subreddits, ensuring coverage of multiple game communities and languages. This includes both general subreddits (e.g., r/gaming or r/games for broad trends) and specific game/franchise subreddits (for deep dives into particular communities). For the multilingual aspect, we will identify games that have international fanbases with active non-English subreddits (for example, a popular title might have separate subreddits in Spanish, Chinese, French, etc., or we can use language filters to collect non-English comments in the main sub). Using the Reddit API (and potentially Pushshift for historical data), we will collect posts and comments spanning key phases of each game’s lifecycle. We will focus on the periods around major events: a few months before release (hype phase), the launch period, and significant intervals after release (e.g., 6 months and 1 year post-release to capture long-term community sentiment). Because the study is concerned with emotional tone, we will also pay attention to discussion threads that are likely to bear emotional content – for instance, reaction threads to game announcements, “review” or “first impression” threads after launch, patch note discussions, and any highly upvoted or controversial threads (indicators of strong community emotion). The dataset will thus be a rich chronological corpus of text posts and comments. All data collection will abide by Reddit’s API terms and ethical guidelines (e.g., excluding deleted or private content, anonymizing usernames) to ensure compliance and privacy. The result will be a multilingual dataset of gaming discussions suitable for sentiment analysis and qualitative coding.

\subsection{Why emotion is important}
 \subsubsection{Emotions Relate to Online Multiplayer Games}
Players in multiplayer video games experience a wide range of emotions. Players experience excitement and joy during victories, and frustration and anger when defeated. These emotional responses are fundamental to the gaming experience \parencite{cuerdo_exploring_2024}. Research shows that video games can evoke emotions spanning exhilaration, pride, empathy, and even sadness or moral reflection, which proves that emotion is at the heart of how players interpret and enjoy games \parencite{cuerdo_exploring_2024}. More importantly, the social context of multiplayer play can modulate these emotional outcomes. For example, one study found that playing a violent game in multiplayer (whether cooperatively or competitively with others) led to less negative affect—i.e., lower hostility and more positive feelings—compared to playing the same game alone \parencite{m_knox_all_2015}. This suggests that having a partner or team can buffer the emotional stress of challenging game content, making the experience feel “safer” or less emotionally detrimental for players \parencite{m_knox_all_2015}. Similarly, having a partner or teammate in-game can buffer the emotional stress of challenging content and make the experience feel “safer” or less emotionally detrimental for players \parencite{liu_connecting_2024}. Co-players not only feel happier, but also reported greater companionship and emotional support, it underline how shared play can enhance emotional benefits and social comfort in game \parencite{liu_connecting_2024}. On the other hand, games that contains intense competition and interactions with opposing online players can also cause a negative emotion during the gameplay, and eventually cause toxic behavior. Studies of online team games (e.g. League of Legends) show that toxic emotional outbursts are contagious among players – if one teammate behaves in a hostile or toxic manner, it can spread to others, undermining the group’s experience \parencite{liu_connecting_2024}. This contagion effect illustrates that emotions in multiplayer settings don’t just reside within one player; they spread through social interactions, affecting team dynamics and overall enjoyment \parencite{liu_connecting_2024}. Because emotions run high in competitive play, researchers have also begun examining how players regulate their feelings – for instance, the gaming community refers to “tilt” as a state of losing control of one’s emotions (often anger or frustration), which is known to impair game performance if not managed \parencite{cregan_playing_2024}. Understanding these emotional experiences and regulating them is critical, as uncontrolled negative emotions can diminish the player’s performance and enjoyment, whereas positive emotions and camaraderie can greatly enhance motivation, satisfzaction, and continued engagement in multiplayer games \parencite{wu_how_2025}. This makes the study of emotional experiences a key pillar for understanding player behavior and well-being in multiplayer contexts.

\subsubsection{Theoretical Frameworks I May Use}
\begin{enumerate}
  \item \textbf{Uses and Gratifications Theory: }This suggests that players are motivated to play games to fulfill certain needs – including emotional needs like entertainment, mood management, and escapism \parencite{ruggiero_uses_2000, huang_emotional_2024}. 
  \item \textbf{Affective or Emotional Design Theory: }Examines how game design elements trigger emotional responses and how those responses influence player behavior \parencite{huang_emotional_2024, jiang_methodology_2015, reynolds_designing_2001}.
  \item \textbf{General Aggression Model (GAM): }Understand how in-game violence and social play impact aggression and affect.\parencite{shaver_general_2011}
  \item \textbf{Social Identity Model of Deindividuation (SIDE): }This sociological lens helped reveal that team-level factors (like a team’s losing status or internal skill disparities) can trigger deindividuation and toxic emotional expressions, and that such toxicity can spread socially (emotional contagion) within teams \parencite{reicher_social_1995}. 
  
\end{enumerate}


