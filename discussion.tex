
\section{Discussion}

This study applies Bowman’s Interactive Demand Framework to naturalistic player discourse to examine how different forms of interactive demand are articulated in relation to enjoyment and frustration \parencite{vorderer_interactivity_2021}. Rather than testing causal effects or evaluating design interventions, the analysis documents patterns in how players describe demand-related strain and recovery in competitive gaming contexts. The discussion interprets these patterns in relation to existing theory and considers their implications and limitations.

\subsection{Empirical Validation of Interactive Demand Framework }

This investigation represents one of the first computational analyses to examine the Interactive Demand Framework using naturalistic player discourse (N = 1,457 texts). By analyzing how players naturally discuss their gaming experiences without researcher prompting or predetermined response categories, this study provides ecological validation of the framework's conceptual structure. The analysis reveals that all five demand dimensions are empirically detectable and meaningfully present in player discourse:

\begin{enumerate}
    \item \textbf{Cognitive Demands} (17.4\% of posts): Players spontaneously discuss strategic thinking, problem-solving, decision-making, and mental challenge. Common linguistic markers include terms such as "think," "strategy," "mode," "level," and "challenge," which indicate natural engagement with the cognitive dimensions of gameplay.
    \item \textbf{Emotional Demands} (17.1\% of posts): Players extensively articulate affective experiences, including excitement, frustration, satisfaction, and emotional intensity. The near-equivalent prevalence to cognitive demands (17.4\% vs. 17.1\%) suggests these two dimensions are psychologically intertwined in player experience, a pattern that is consistent with the framework’s emphasis on the interdependence of cognitive and emotional demands.
    \item \textbf{Social Demands} (15.2\% of posts): Discussions of teamwork, community, multiplayer coordination, and social dynamics demonstrate that social dimensions are central to contemporary gaming discourse. The prominence of platform names like "Steam" and social terms like "friend," "team," and "community" indicates that social demands extend beyond in-game mechanics to encompass broader gaming social ecosystems.
    \item \textbf{Physical Demands} (6.5\% of posts): While less frequently discussed, physical comfort, ergonomics, and bodily engagement emerge as discrete concerns in player discourse. The relatively lower frequency suggests physical demands may function as "hygiene factors" in gaming experience—assumed baseline conditions that become salient primarily when violated.
    \item \textbf{Controller Demands} (4.9\% of posts): Input responsiveness, control schemes, and interface design are the least frequently discussed dimensions but, critically, show the most extreme negative sentiment when mentioned (80.6\% negative valence). This pattern suggests controller demands operate similarly to physical demands as hygiene factors, but with even more severe consequences when problematic.
\end{enumerate}

The spontaneous emergence of all five demand dimensions in player-generated discourse, without researcher prompting or imposed frameworks, provides strong convergent validity for Bowman's conceptualization. Players do not use academic terminology like "cognitive load" or "emotional regulation," yet they naturally express these constructs through gaming-specific language. This demonstrates that the framework captures phenomenologically real aspects of player experience rather than researcher-imposed artifacts. Furthermore, the relative frequencies of demand mentions (cognitive and emotional dominating, followed by social, with physical and controller trailing) reveal a hierarchy of psychological salience that was not explicit in the original framework but emerges from player priorities. This distribution pattern suggests that cognitive and emotional demands are core to the gaming experience, social demands are increasingly important in contemporary multiplayer contexts, and physical/controller demands function as necessary but not sufficient baseline conditions.

Taken together, these findings suggest that interactive demands in multiplayer games are rarely experienced in isolation. Instead, players’ accounts point to moments in which multiple demands converge, creating what may be understood as a compound demand load. Within such moments, enjoyment and strain are not mutually exclusive, but coexist in a fragile balance that can shift rapidly depending on contextual factors such as team coordination, interface reliability, and perceived fairness of challenge. This perspective extends the Interactivity-as-Demand framework by emphasizing not only the presence of individual demand dimensions, but also their co-occurrence and interaction in lived play experiences. Rather than treating cognitive, emotional, social, physical, and controller demands as parallel contributors to player experience, the present findings suggest that their alignment or misalignment may be critical in shaping when gameplay feels engaging versus overwhelming.

Importantly, this analysis does not claim to capture players’ internal emotional states directly. Instead, it illuminates how players publicly articulate and negotiate these demand experiences within online discourse. As such, the results should be understood as mapping the expressive contours of gameplay-related strain and enjoyment, rather than providing a comprehensive account of psychological outcomes. These considerations motivate both the limitations of the present approach and several directions for future research.

\subsection{Implications for Game Designers and Developers}
\subsubsection{Cognitive Demand Design}

The analysis shows that cognitive demands are the most frequently discussed demand type (17.4\% of posts) and are associated with both enjoyment and frustration. Players often describe cognitively demanding gameplay as satisfying when challenges are perceived as fair and learnable, but frustrating when difficulty spikes feel arbitrary or excessive. This pattern is consistent with Flow theory, which suggests that enjoyment emerges when challenge is well matched to perceived skill, while frustration arises when this balance collapses \parencite{cziksentmihalyi_flow_1990}. The present findings do not test flow states directly, but show how players articulate experiences that resemble this balance in everyday discourse. This pattern suggests that cognitive demand is central to player experience but highly sensitive to calibration.


The finding that cognitive demands constitute the most frequently discussed dimension (17.4\% of all posts), accompanied by moderate positive sentiment (30.0\%) and substantial negative sentiment (67.7\%), has significant implications for difficulty design. This pattern indicates that cognitive challenge is central to the gaming experience, yet it is highly sensitive to calibration. The recurring use of the phrase "challenging but fair" among players underscores a core design principle that aims to strike a balance between difficulty and perceived fairness. Moderate positive sentiment suggests that appropriately calibrated cognitive demands promote satisfaction, accomplishment, and engagement, which are characteristic of flow states, as described by \textcite{cziksentmihalyi_flow_1990}. However, the predominance of negative sentiment reveals the challenges and frequency of failure in achieving optimal calibration. Players often report frustration when cognitive demands are excessively high ("impossible difficulty spike"), insufficiently challenging ("boring and repetitive"), or perceived as unfair ("fake difficulty through poor design").These findings yield several actionable design recommendations. First, adaptive difficulty systems, which automatically adjust game challenge based on player performance, should be prioritized as core features rather than optional enhancements. Second, difficulty curves should undergo extensive play testing across a broad spectrum of player skill levels, as what appears balanced to expert developers may be excessively challenging for average players. Third, transparency in difficulty mechanics, achieved by providing clear information about how and why challenges are structured, enables players to differentiate between personal skill limitations and design flaws. Providing explicit feedback regarding failure and strategies for improvement can convert frustrating cognitive demands into motivating challenges. Additionally, the strong correlation between cognitive and emotional demands (r = 0.190) demonstrates that cognitive design decisions have direct emotional consequences. Designers should evaluate not only the mechanical solvability of puzzles or combat encounters but also the emotional experience associated with the problem-solving process. Implementing incremental progress markers, partial success rewards, and clear learning trajectories can help ensure that cognitively demanding content elicits positive affect even before full mastery is achieved. 

\subsubsection{Controller Demand Design}
Controller demands appear infrequently in player discourse (4.9\% of posts), yet when mentioned they are associated with the highest proportion of negative sentiment (80.6\%). This pattern aligns with Norman’s observation that effective interfaces tend to disappear from conscious attention, while poorly designed controls draw immediate frustration and blame \parencite{norman_design_2013}. This dynamic resembles what Herzberg described as hygiene factors: baseline conditions that do not increase satisfaction when present but produce dissatisfaction when absent \parencite{herzberg_work_1966}. The discourse suggests that controller reliability functions in this way, as smooth input rarely generates praise, while failures override enjoyment in other demand dimensions. In this sense, controller demands operate as interaction conditions that become salient primarily when they fail. This pattern indicates that players rarely comment on control systems when they function smoothly, but strongly react when interface or input issues disrupt gameplay. In this sense, controller demands function as baseline conditions whose failure can override enjoyment in other areas.

Games that deprioritize controller performance in favor of other features risk severe player dissatisfaction, regardless of strengths in other areas. As one player stated: "The game could have the best story ever, but if the controls suck, I'm not playing it."Interface design should emphasize transparency and learnability. \textcite{norman_design_2013} principles of good design are not merely aesthetic considerations but are fundamental to player engagement with core game content. Complex games, such as strategy, role-playing, and simulation titles, face unique challenges in presenting intricate mechanics without overwhelming users. Iterative usability testing with diverse player populations, including those unfamiliar with genre conventions, is essential for identifying interface friction points that may be overlooked by developers. Accessibility and customization options are not merely inclusive design choices, but essential quality-of-life features for all players. Features such as re-mappable controls, adjustable sensitivity, customizable UI scaling, and colorblind modes accommodate the diverse physical capabilities, gaming experience, and preferences of players. The pronounced frustration regarding controller issues in the data suggests that standardized input schemes may alienate portions of the player base.Platform-specific optimization, which involves tailoring controls and interfaces for different devices, is crucial because controller demands vary across PC, console, and mobile platforms. The frequent reference to "Steam" in controller-related discussions suggests that PC players are particularly sensitive to technical performance, likely due to hardware variability and the precision required by mouse and keyboard interfaces. Console players prioritize controller ergonomics and button layout, while mobile players focus on touch responsiveness and screen space. Cross-platform games should adapt their input schemes to each platform rather than relying on generic solutions that fail to fully meet the needs of any platform. 

\subsubsection{Social Demand Design}
Social demands exhibit a polarized sentiment profile, with both high positive and high negative expression. Players describe positive social experiences in terms of teamwork, coordination, and shared success, while negative social experiences often involve toxicity, blame, or interpersonal breakdown. The co-occurrence of social demands with cognitive and emotional demands suggests that social interaction intensifies other forms of demand rather than operating independently. The observation that social demands exhibit the highest positive sentiment (33.8\%) alongside substantial negative sentiment (63.5\%) indicates that social features are highly variable and polarizing elements within multiplayer gaming. Effective social interactions, such as finding compatible teammates, experiencing positive community dynamics, and engaging in coordinated play, are closely tied to peak gaming experiences. In contrast, failures in social interaction produce intensely negative experiences that may contribute to player attrition. The strong co-occurrence of social demands with both cognitive (r = 0.112) and emotional demands (r = 0.137) indicates that social features are deeply integrated with core gameplay experiences rather than functioning as isolated mechanics. Strategic coordination merges cognitive challenge with social interaction, while shared victories and defeats intensify emotional responses through social comparison and collective efficacy. Consequently, social features should be designed as fundamental components of the player experience ecosystem, not as optional add-ons.

These patterns inform several design recommendations for promoting positive social experiences and reducing toxicity. Robust moderation systems are essential infrastructure for any game featuring social components, rather than optional community management tools. The data indicate that toxic behavior, griefing, and harassment are significant drivers of negative sentiment that undermine the social experience. Effective moderation should encompass multiple layers, including automated detection of offensive language and disruptive behavior, player reporting mechanisms with responsive enforcement, and community guidelines that clearly define behavioral expectations and consequences. Positive reinforcement systems, such as League of Legends' Honor system or Overwatch's endorsement system, can complement punitive measures by rewarding prosocial behavior. Matchmaking systems should account for both player skill and behavioral preferences. The frequent reference to "toxic teammates" in negative social discourse suggests that skill-based balance alone is insufficient if behavioral incompatibility persists. Reputation systems that monitor communication style, cooperation, and sportsmanship can facilitate the matching of compatible players. Furthermore, offering multiple play modes, including competitive ranked play, casual unranked modes, and single-player or bot options, acknowledges the diverse social preferences among players.

Communication tools should be designed to support positive coordination while minimizing opportunities for harassment. While voice chat facilitates tactical coordination, it also introduces risks of verbal abuse; similarly, text chat enables strategic discussion but can be misused. Quick message systems, such as Rocket League's preset phrases, allow communication while reducing the potential for toxicity. The optimal communication design varies by genre, audience, and cultural context, but all systems should include muting, blocking, and reporting as standard features. Games targeting younger audiences or those with histories of toxic communities may benefit from more restrictive communication options, supplemented by context-sensitive quick messages. Social features should accommodate both interactions among strangers and established friend groups, as these contexts generate distinct social demands. Random matchmaking with strangers introduces coordination challenges but fosters social discovery, while playing with friends offers psychological safety but necessitates group management. Features such as clan systems, friends lists, party voice chat, and options for solo or team queues enable players to select their preferred level of social engagement. Community features that extend beyond in-game interactions can enrich the social dimension of gaming and promote a positive culture. The frequent mention of platform names, such as "Steam," and community spaces, like subreddits, in the data indicates that the gaming social experience persists beyond gameplay sessions. Developers who engage with these communities, support content creators, organize events, and cultivate inclusive cultures can positively influence the social demand experience outside of direct gameplay. 

\subsubsection{Physical Demand Design}
Although physical demands are discussed less frequently (6.5\% of posts), they are associated with high negative sentiment (74.5\%) when mentioned. Player discourse emphasizes fatigue, discomfort, and strain as barriers to sustained engagement rather than sources of enjoyment. This pattern suggests that physical demands, similar to controller demands, become salient primarily when they disrupt play.

Design recommendations for mitigating physical demands emphasize flexibility and accessibility. Control schemes should minimize repetitive strain by avoiding button-mashing mechanics, offering toggle options instead of hold-to-sprint, and designing combat systems that prioritize timing and positioning over input speed. Titles such as Dark Souls illustrate that strategic, deliberate combat can be both engaging and less physically demanding than reflex-based systems. Visual design should enhance eye comfort through adjustable brightness, contrast, and gamma settings, as well as options to reduce motion blur, screen shake, and camera shake. Colorblind-accessible palettes should also be provided. Pacing mechanisms, including natural breakpoints, session length warnings, and game play loops that encourage varied activities, can help prevent fatigue during extended sessions. Finally, supporting alternative control schemes ensures that physical limitations do not restrict access to content.

The increasing prominence of VR gaming heightens the importance of addressing physical demands, as VR introduces vestibular, kinesthetic, and proprioceptive factors not present in traditional gaming. VR developers should carefully consider comfort settings, locomotion options, session pacing, and physical space requirements to minimize the risk of motion sickness and fatigue. Lessons from traditional gaming regarding physical demands as hygiene factors are especially pertinent in VR, where these issues are magnified. 

\subsubsection{Emotional Demand Design}
The observation that emotional demands exhibit mixed valence (33.3\% positive and 65.5\% negative) indicates that emotional engagement in gaming is complex and multifaceted, rather than solely focused on maximizing enjoyment. This finding is consistent with research on meaningful media experiences, which demonstrates that games, like films and literature, can deliver value through a spectrum of emotions, including sadness, fear, anger, and moral discomfort (Oliver and Bartsch, 2010). The strong correlation between emotional and cognitive demands (r = 0.190) further indicates that emotional responses are integrated with intellectual engagement. These findings challenge the assumption that game design should prioritize maximizing positive affect and minimizing negative affect. Instead, they suggest that emotional richness and intensity may be more valuable than a singular focus on positivity. Games that pursue artistic expression, narrative depth, or moral complexity should strive to encompass the full range of human emotions, rather than limiting themselves to hedonic pleasure.This perspective has implications for narrative design, mechanical feedback, and moment-to-moment pacing. Narrative-driven games can incorporate themes such as loss, sacrifice, moral ambiguity, and emotional vulnerability without concern that negative emotions will deter players; these themes may, in fact, foster more memorable and meaningful experiences. Mechanical design can employ emotional pacing by alternating tension and relief, challenge and mastery, and setbacks and triumphs to create dynamic affective trajectories. Visual and audio design can utilize atmosphere, music, and aesthetic choices to evoke specific emotional tones, ranging from the sustained dread of horror games to the bittersweet reflection of melancholic indie titles. However, the data also indicate limits to players' tolerance for negative affect. While negative emotions are accepted as part of engaging experiences, extreme or prolonged negativity without resolution leads to dissatisfaction. Frustration from difficulty is acceptable when it results in mastery and accomplishment, and fear in horror games is enjoyable because players can control their exposure and seek relief. Emotional design should therefore incorporate agency, pacing, and resolution, rather than subjecting players to a continuous stream of negativity. The balance between emotional intensity and player control determines whether negative emotions are perceived as meaningfully challenging or merely punitive. 

\subsection{Limitations and Future Directions}
\subsubsection{Methodological and Data Limitations}
This study relies on Reddit discussions to infer players’ emotional and cognitive experiences in multiplayer FPS games. While Reddit provides rich, naturally occurring discourse, it represents a selective population. Users who post are more likely to be highly engaged, dissatisfied, or motivated to articulate strong opinions. Silent players, casual players, and those who disengage without posting remain largely invisible. As a result, the observed demand patterns likely reflect heightened or polarized experiences rather than the full distribution of player responses. The platform context also shapes expression. Reddit encourages narrative framing, humor, and exaggeration, which can amplify perceived frustration or enjoyment. Emotional language on Reddit should therefore be interpreted as communicative performance rather than direct psychological measurement. This limitation constrains claims about prevalence or intensity of stress at the population level. The findings instead speak to how players \textit{talk about} demands, not how frequently those demands are objectively experienced.

A second limitation concerns the operationalization of interactive demands through keyword-based and model-assisted text classification. Although multiple sentiment methods were triangulated, agreement across methods remained moderate rather than high. This divergence highlights a broader issue in computational text analysis: emotional nuance, irony, and context-specific meanings are difficult to capture through automated approaches alone. Keywords may overcount repetitive expressions, while transformer-based models may misinterpret gaming slang or genre-specific norms.

Finally, the analysis is cross-sectional. Posts are treated as independent observations, even though many users discuss their experiences across time. Without longitudinal user-level tracking, it is not possible to examine how emotional demand accumulates, resolves, or transforms across repeated play sessions. Stress trajectories, burnout processes, and recovery patterns therefore remain inferred rather than observed.


\subsubsection{Conceptual Scope and Theoretical Constraints}
The Interactivity-as-Demand framework provides a productive lens for organizing player discourse, yet its application here remains descriptive. Demands are identified and compared, but their causal ordering is not tested. For example, co-occurrence between cognitive and emotional demands does not establish whether cognitive overload produces emotional frustration, or whether negative affect heightens perceived difficulty. These relationships are theoretically plausible but empirically unresolved within the present design. In addition, the framework treats demand dimensions as analytically separable, while player discourse often blends them. Physical strain, controller issues, and cognitive effort frequently appear within a single complaint. This overlap suggests that demands may operate as configurations rather than isolated dimensions. Treating them independently risks oversimplifying how players experience gameplay as an integrated system of constraints. The study also centers on stress and enjoyment, leaving other meaningful outcomes underexplored. Players often reference identity, self-efficacy, and social belonging implicitly. These constructs are theoretically adjacent but not explicitly modeled. As a result, the emotional consequences of demand imbalance are likely broader than the analytic categories employed here.

\subsubsection{Future Directions for Research Design}
Future work should extend this computational approach through mixed-method integration. Large-scale text analysis is well suited for identifying patterns, but it should be paired with surveys or experience sampling to validate psychological interpretations. Linking discourse-derived demand indicators with self-reported stress, enjoyment, or recovery would strengthen construct validity and allow for individual-level inference. Longitudinal designs represent a critical next step. Tracking discussions across patches, rank changes, or seasonal resets would enable analysis of how demand perceptions evolve over time. Such designs could distinguish short-term frustration from chronic burnout, a distinction that remains blurred in cross-sectional snapshots. Methodologically, future research should move beyond keyword detection toward demand-aware classifiers trained on game-specific corpora. Human-annotated training data that reflect gaming vernacular would improve sensitivity to context and reduce misclassification. Rather than seeking a single “best” sentiment model, future pipelines should explicitly model disagreement across methods as substantive information rather than error.

Beyond methodological refinement, future research should aim to translate demand analysis into design-relevant insights. The present findings suggest that frustration often emerges not from high demand alone, but from perceived misalignment between effort and reward. This observation motivates a shift from demand intensity to demand calibration as a design principle. A promising direction involves modeling demand–emotion imbalance as a dynamic process. When demands exceed players’ adaptive capacity, enjoyment collapses into stress. When demands are well matched, challenge becomes meaningful rather than exhausting. Formalizing this balance would allow researchers to generate testable predictions and designers to identify intervention points.

Finally, extending this framework beyond FPS games would clarify its generalizability. Competitive pressure, social dependence, and mechanical precision are especially salient in FPS contexts. Other genres may reveal different demand hierarchies. Comparative analysis across genres would help determine which patterns are genre-specific and which reflect broader properties of interactive media.
