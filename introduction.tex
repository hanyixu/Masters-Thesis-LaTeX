
\newpage


% \begin{spacing}{1}
% \noindent\fbox{%
%     \parbox{\dimexpr\textwidth-2\fboxsep-2\fboxrule\relax}{
%         \textbf{Writing Sample}\\
%         Hanyi Xu \,|\, xuhanyi2025@gmail.com\\[6pt]
%         Excerpt from Master's Thesis: Why Games Stop Feeling Fun: A Computational Approach to Emotional Demand in Play 
%         (Submitted to the Department of Quantitative Methods in the Social Sciences, Columbia University in the City of New York. 
%         In Fulfillment of the Requirements For the Degree of Master of Arts)\\[6pt]
%         Advisors: Dr.~Gregory Kossinets, Dr.~Elena Krumova
%     }
% }
% \end{spacing}
\section{Introduction}

\subsection{Background of the Study}

Video gaming is one of the most widespread forms of entertainment in the United States. According to the 2025 Essential Facts report by the Entertainment Software Association (ESA), 205.1 million Americans between ages 5 and 90 play video games weekly, up from 61\% in 2024 \parencite{association_2025_2025}. The average gamer is 36 years old, suggesting that gaming is a hobby enjoyed by people across various age groups. Additionally, 60\% of adults (18+) now play weekly, with 49\% of Baby Boomers and 36\% of the Silent Generation participating in regular play \parencite{association_2025_2025}. The COVID lockdowns accelerated this trend as people turned to games for recreation and social connection. This player base offers a valuable context for examining the impact of gaming on society and mental health. These changes suggest the medium's evolution from niche entertainment to a mainstream, inter-generational practice, positioning gaming as a subject of psychological and social research.

Building on these national trends, the past decade has seen the global gaming population expand at an unprecedented rate, reflecting the growing cultural and economic influence of interactive entertainment. According to \textcite{newzoo_global_2024}, more than 3.42 billion people worldwide now play video games, marking a 4.5 percent increase from 2023 and demonstrating the medium’s deep integration into everyday life . This surge is driven by the convergence of accessibility, technology, and social connectivity: smartphones have made gaming ubiquitous across age and income groups, while cross-platform play, streaming, and online communities have transformed gaming into a shared cultural experience rather than a solitary pastime. Emerging markets are experiencing the fastest growth, as mobile connectivity and affordable hardware continue to lower entry barriers. In the United States and China, which together account for nearly half of global consumer spending, gaming has matured into a dominant entertainment sector rivaling film and television in both revenue and audience engagement \parencite{newzoo_global_2024}. This global spike underscores a broader shift in media consumption, where interactivity, personalization, and social participation redefine how individuals seek leisure, identity, and connection in the digital age \parencite{newzoo_global_2024}.


Unlike traditional media, electronic games are highly interactive.Interactive media like games and VR give users "lived experiences" in rich, immersive contexts, which can measurably shape how people think, feel, and behave in real life \parencite{bowman_paradox_2020}. In other words, because players make choices and control the action, games can engage them more deeply than non-interactive media, leading to stronger cognitive and emotional involvement. This further suggests that player engagement with games can be beneficial in moderation or harmful in excess, and recent work in psychology underscores this delicate balance. 

Research in psychology and human-computer interaction suggests that video games can be both an enjoyable pastime and a means of emotional management. Playing games can relax, distract, and alleviate stress. For example, an interview study conducted during the pandemic reported that video games provided psychological benefits to players under stress, helping them cope with daily anxieties. These findings align with narratives among gamers who use games to relax or cope with negative emotions.

To operationalize this concept of effortful engagement, this study relies on Interactive Demand Framework (IDF) \parencite{vorderer_interactivity_2021}, which conceptualizes video games as “demanding technologies” distinct from passive media. Bowman posits that games impose continuous requirements on the player, necessitating active investment across cognitive, emotional, social, and physical dimensions to sustain the experience. This framework provides the essential vocabulary for understanding how players negotiate these pressures, suggesting that it is precisely through the satisfaction of these demands that the “labor” of play transforms into immersive co-creation.

\subsection{Statement of the Problem}

Research methodologies investigating player experience predominantly utilize quantitative surveys and controlled environments. For instance, a systematic review of affect-adaptive games found that nearly three-quarters of studies employed self-assessment instruments to evaluate user experience \parencite{croissant_theories_2023}. However, data obtained from self-report surveys often lack naturalistic, player-driven perspectives on how individuals discuss and interpret their gaming experiences. Moreover, requiring participants to recall specific game moments in questionnaires may introduce bias or fail to capture the core issues that the survey targets. The article also notes that most studies reviewed had insufficient sample sizes and did not conduct statistical power analyses. Consequently, many findings exhibited limited generalizability, and results indicating "no significant difference" may reflect inadequate statistical power rather than a true absence of effect \parencite{croissant_theories_2023}. 

A notable gap exists in naturalistic, player-driven data that captures how gamers authentically describe the demands of gameplay. Online communities, such as Reddit, provide a valuable corpus. Players frequently express frustration, accomplishment, immersion, and emotional fatigue there. These posts facilitate an analysis of how the theoretical constructs of the Interactive Demand Framework are reflected in everyday conversation. This study addresses the gap by examining how players describe and negotiate interactive demands in real-world online contexts. By analyzing Reddit discussions, the research maps the cognitive, emotional, physical, social, and controller demands mentioned by players. It also seeks to determine how these discussions correspond with or extend Bowman’s conceptual framework.

\subsection{Research Question}

Guided by the gaps identified in the literature and grounded in Interactive Demand Framework, this study aims to explore how players describe and interpret the interactive demands of video games in online discussions. To achieve this, the research is guided by the following questions:
\begin{enumerate}
    \item How do Reddit users articulate and describe the cognitive, emotional, physical, social, and controller demands experienced during video game play?
    \item Which types of interactive demands are most frequently discussed across Reddit gaming communities?
    \item In what ways do players express enjoyment, frustration, or balance in relation to these demands?
    \item How do patterns in Reddit discussions reflect, reinforce, or extend Bowman’s Interactive Demand Framework?
\end{enumerate}



