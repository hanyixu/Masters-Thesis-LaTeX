
\newpage


% \begin{spacing}{1}
% \noindent\fbox{%
%     \parbox{\dimexpr\textwidth-2\fboxsep-2\fboxrule\relax}{
%         \textbf{Writing Sample}\\
%         Hanyi Xu \,|\, xuhanyi2025@gmail.com\\[6pt]
%         Excerpt from Master's Thesis: Why Games Stop Feeling Fun: A Computational Approach to Emotional Demand in Play 
%         (Submitted to the Department of Quantitative Methods in the Social Sciences, Columbia University in the City of New York. 
%         In Fulfillment of the Requirements For the Degree of Master of Arts)\\[6pt]
%         Advisors: Dr.~Gregory Kossinets, Dr.~Elena Krumova
%     }
% }
% \end{spacing}
\section{Introduction}

\subsection{Background of the Study}

Video games are commonly used for relaxation, mood repair, and social connection. Yet many players report a different outcome. After a gaming session, they may feel stressed, frustrated, or emotionally drained. This paradox is especially visible in online multiplayer environments, where competition, coordination, and social evaluation intensify emotional responses \parencite{cuerdo_exploring_2024}. Multiplayer play can amplify joy during victories, but it can also magnify anger, tension, and interpersonal conflict after losses. These emotions do not remain confined to individual players. They spread through communication and shared performance, shaping collective experience \parencite{liu_connecting_2024, lee_less_2025}. In competitive communities, players often refer to “tilt” to describe a loss of emotional control that impairs both performance and enjoyment \parencite{cregan_playing_2024}. At the same time, positive emotions and camaraderie can enhance motivation, satisfaction, and continued engagement \parencite{wu_how_2025}. Together, these findings raise a central question. When does gaming support emotional recovery, and when does it instead become emotionally taxing?

This question matters because gaming is no longer a niche activity. A substantial proportion of adults in the United States play video games regularly across age groups \parencite{association_2025_2025}. Globally, gaming has become a routine part of everyday media consumption, with billions of players worldwide \parencite{newzoo_global_2024}. As gaming becomes embedded in daily life, its emotional consequences carry broader implications for well-being, social interaction, and media use.

A key challenge in addressing this question is that video games are not passive media. Games require sustained attention, continuous input, and rapid adjustment to feedback. This thesis adopts the Interactivity-as-Demand framework, which conceptualizes games as demanding technologies rather than effortless entertainment \parencite{vorderer_interactivity_2021}. The framework distinguishes multiple forms of demand, including cognitive, emotional, physical, and social demands. These demands often operate simultaneously. Because attention and self-regulatory resources are limited, demands can compete. This competition helps explain why an experience that initially feels engaging can later feel overwhelming or exhausting.

Flow theory provides an additional foundation for understanding this shift \parencite{cziksentmihalyi_flow_1990}. Flow emerges when challenge matches skill and feedback is clear. When challenge exceeds skill, frustration and stress increase. When challenge is too low, boredom emerges. Flow theory therefore explains when demands are experienced as satisfying effort rather than excessive strain. Mood Management Theory further suggests that people select media to regulate emotion and reduce stress \parencite{reinecke_mood_2016}. Uses and Gratifications research similarly emphasizes that players enter games with expectations, such as relaxation, achievement, or social connection \parencite{ruggiero_uses_2000}. When outcomes fail to meet these expectations, negative emotions can intensify. Together, these perspectives converge on a central idea. Emotional outcomes in gaming depend on the fit between player resources, player expectations, and the demands imposed by play.

\subsection{Statement of the Problem}

Despite extensive research on player emotion, most studies rely on surveys or controlled laboratory designs. These approaches provide valuable insight, but they often capture outcomes rather than processes. Self-report measures require recall, which can distort moments of peak frustration, exhaustion, or emotional breakdown. A recent review of affect-adaptive game research highlights heavy reliance on self-assessment instruments, along with limitations related to sample size and statistical power \parencite{croissant_theories_2023}. As a result, many studies offer limited insight into how emotional strain accumulates during play and how players articulate demand-related experiences in their own words. This thesis addresses this gap by analyzing naturally occurring player discourse from Reddit gaming communities. Reddit provides access to spontaneous, player-generated accounts of frustration, enjoyment, burnout, and social conflict. These discussions offer high ecological validity for examining how players publicly narrate their experiences. At the same time, Reddit data are inherently biased. Reddit users are self-selected, more vocal, and not representative of the broader gaming population. Findings from this study should therefore be interpreted as patterns of discourse and perceived demand, not as population-level prevalence estimates. Rather than replacing traditional surveys or experiments, Reddit-based analysis complements them by revealing how interactive demands are framed, negotiated, and emotionally evaluated in everyday talk.

To analyze these narratives, this study integrates computational methods with theory-driven demand coding. This approach aligns with a broader movement toward computational modeling in psychological research, which aims to make implicit cognitive and emotional processes explicit through formal analysis \parencite{stein_computational_2025}. Related work in game-based learning also treats play as a dynamic system shaped by task structure, feedback, and repeated attempts \parencite{westera_how_2017}. Building on these ideas, this thesis conceptualizes player experience as a continuously updating cognitive-affective system that evolves through interaction.