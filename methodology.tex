%\newpage
\section{Methodology}

\subsection{Research Design}

To ensure the reliability and reproducibility of the analysis results, this study will use multi-level rigor checks. During text sentiment recognition, the DistilBERT model will be fine-tuned and validated. Cross-comparison with manually annotated sample subsets will assess the consistency of predictions and classification accuracy. For topics generated by BERTopic or LDA, noise sensitivity tests and topic consistency metrics, such as coherence scores, will assess the stability of model output. The study will also investigate how different parameter settings and embedding methods impact the topic structure. In the interactive requirement coding framework, clear annotation rules and mutual annotation processes will ensure consistency and accuracy. Cohen’s Kappa will measure annotation reliability, providing an operational basis for judging requirement categories. By incorporating statistical validation, annotation consistency, and model robustness analysis, this study presents an analytical framework to ensure future results are rigorous, reproducible, and scalable.

This thesis will use an observational and compuational social sciecne design to analyze the natural occurring Reddit posts. As the data is derived from publicly available discussions rather than researcher-generated responses, the study does not involve experimental manipulation or intervention. The research seeks to capture large-scale language patterns and nuanced emotional themes by integrating quantitative natural language processing methods with qualitative topic coding. Reddit serves as an effective platform for this analysis because users often provide candid accounts of their gaming experiences, including emotions such as stress, frustration, burnout, and loneliness. The primary objectives are to identify sources of stress among players, uncover recurring emotional patterns, and examine personality-related signals present in gaming discourse.

\subsection{Data Source: Reddit}

This thesis collects data from Reddit, a highly anonymous, discussion-based social platform. Users typically share gaming experiences, emotional reactions, and social interactions through long-form posts. Compared to questionnaires, expressions on Reddit may be more natural and may better reflect actual gaming experiences. For this study, self-written Python code was used to systematically collect posts and comments.

First, this study identifies primary discussion communities related to FPS games. Based on designated program parameters, a crawler retrieves posts from several prominent FPS game communities, namely r/VALORANT, r/Overwatch, r/GlobalOffensive (CS:GO/CS2), r/apexlegends, r/CallOfDuty, r/competitivegaming, r/gaming, and r/truegaming. These communities are highly active on Reddit, with user bases primarily comprising young players fluent in FPS game culture, which aligns with the study's objectives. Regarding data collection, the study emphasizes 'stress experience' and 'motivational fluctuations.' The pipeline deploys groups of keywords to filter for posts concerning stress (e.g., 'stress,' 'tilted,' 'burnout'), motivation (e.g., 'fun,' 'enjoy,' 'quit playing'), and social experiences (e.g., 'team,' 'solo,' 'lonely').

Each subreddit is searched for two sets of keywords: one for stress and another for motivation and enjoyment. By default, the program scrapes about 50 relevant posts from each community (25 for stress, 25 for motivation) and approximately 10 comments per post, using post IDs, to create a comprehensive interaction data package. Simultaneously, the crawler records basic metadata for each post, including posting time (UTC), post score, comment count, self-post status, upvote ratio, author information, keyword frequency, and whether the post mentions behavioral clues such as team or solo modes. This information supports the analysis of interaction pressure, social needs, emotional patterns, and differences in game modes.
\begin{table}[ht]
\centering
\caption{Data Collection Overview}
\label{tab:data_collection_overview}

\begin{tabular}{|>{\raggedright\arraybackslash}p{3.2cm}|
                >{\raggedright\arraybackslash}p{12.3cm}|}
\hline
\textbf{Item} & \textbf{Description} \\
\hline\hline

Data Source & Reddit (public posts and comments) \\
\hline
Collection Method & Automated Python pipeline using the Reddit API \\
\hline
Target Communities (Subreddits) & r/VALORANT, r/Overwatch, r/GlobalOffensive, r/apexlegends, r/CallOfDuty, r/competitivegaming, r/gaming, r/truegaming \\
\hline
Collection Strategy & Keyword-based retrieval of stress-related and motivation-related posts \\
\hline
Stress Keywords & stress, stressed, tilted, burnout, anxious, anxiety, frustrated, overwhelmed, pressure \\
\hline
Motivation Keywords & fun, enjoy, motivation, progress, improve, quit playing \\
\hline
Posts per Subreddit & Approximately 50 posts per subreddit (25 stress-related and 25 motivation-related) \\
\hline
Comments per Post & Up to 10 comments collected per post \\
\hline
Search Time Window & One year \\
\hline
Filtering Rules & English only; excludes memes, tech posts, patch notes, and ultra-short posts \\
\hline
Extracted Metadata & Timestamp, score, upvote ratio, comment count, keyword counts, matched keyword type (stress or motivation), game mode indicators (team vs. solo), text length \\
\hline
Processing Tools & Python, PRAW/Pushshift API, Pandas \\
\hline

\end{tabular}

\end{table}

This study applies clear inclusion and exclusion criteria. These methods ensure the dataset captures meaningful emotional and experiential content relevant to gaming stress and motivation. Posts were included if they described personal experiences related to gameplay. This includes emotions such as frustration, loneliness, burnout, or changes in motivation. Posts were included if they mentioned terms like “ranked,” “solo queue,” “burnout,” “anxiety,” or “overwhelm.” These terms are strong indicators of stress or demand-related gaming experiences. Only English-language posts were used to maintain consistency in linguistic analysis. This also ensures compatibility with the NLP tools used later. Posts were excluded if they consisted mainly of memes, jokes, patch notes, hardware discussions, or other technical or comedic content without a player’s emotional experience. Very short posts or those containing mostly links or emojis were also omitted. These posts lacked sufficient text for analysis. These criteria collectively focus the dataset on rich, experience-based narratives, rather than unrelated or low-content material. 

After collection, the dataset underwent extensive cleaning and preprocessing to remove noise and other artifacts. This prepared the text for computational analysis. Deleted posts, bot-generated content, and non-English entries were removed during the initial filtering process. Standard text-cleaning steps included lowercase conversion and removal of URLs, emojis, and HTML artifacts. The text then went through tokenization and sentence segmentation. Stopword removal and lemmatization were also used to normalize linguistic variation. These steps improve the performance of NLP models. Posts with too few words were excluded to ensure enough context for reliable classification. Optional preprocessing included flagging posts that referenced specific games. It also identified whether discussions took place in single-player or multiplayer contexts. These steps collectively produced a clean and consistent dataset suitable for sentiment analysis, topic modeling, and theory-driven text coding. 


\subsection{Analytical Strategy}

\subsubsection{Sentiment Analysis and Topic Modeling}

We employ a DistilBERT-based emotion classification model. DistilBERT is a simplified and faster version of the BERT (Bidirectional Encoder Representations from Transformers) model, which classifies the emotional tone of each Reddit post into categories such as stress, anger, loneliness, frustration, and general negative affect. These emotion labels serve as quantitative indicators of players’ affective responses and facilitate the identification of patterns across different game contexts and demand types. This directly addresses the research questions by clarifying how players describe stress and negative outcomes in authentic gaming conversations.

We apply topic modeling to uncover thematic structures and identify recurring gameplay related stressors using either Latent Dirichlet Allocation (LDA) or BERTopic. BERTopic offers more coherent topic separation through transformer embeddings and hierarchical clustering. Themes include team coordination pressure, exposure to toxic chat, burnout, social disconnection during solo play, and cognitive overload from high-intensity gameplay. Mapping these topics into interactive demand categories enables analysis of how various demand types emerge in players’ narratives and relate to negative emotional experiences.

\subsubsection{Identifying Gratification Mismatch}

To examine the mismatch between players’ intended gratifications and their actual gameplay experiences, we use targeted pattern matching and phrase-level analysis. We identify expressions such as “I play to relax but…,” “I just want fun but…,” or “I play for social interaction but…” to detect unmet expectations. These mismatch expressions are coded into categories, including unmet relaxation needs, unmet social connection needs, and unmet achievement or competence needs. Detecting these discrepancies clarifies when and why games fail to provide their intended psychological benefits and instead generate stress, loneliness, or dissatisfaction. This mismatch framework directly supports the research objective of identifying when interactive demands exceed players’ motivational expectations. We analyze personality-related language by extracting self-descriptive statements in posts, such as “I’m anxious,” “I’m introverted,” “I get stressed easily,” or “I’m competitive.” We categorize posts with these descriptors into personality-coded groups. This lets us examine if individual differences shape stress and motivation patterns. We then compare emotion and topic distributions in personality-coded and non-personality posts to spot any moderating effects. This analysis helps us explore whether some personality traits are linked to higher cognitive pressure, greater emotional volatility, or stronger reactions to multiplayer social dynamics.


Unlike traditional player experience research, which uses questionnaires and controlled experiments, this study uses the Reddit natural language corpus (a large collection of user-generated text from Reddit). This approach replaces self-report measures. It allows us to capture dynamic psychological mechanisms through natural language processing and topic modeling. Conventional approaches rely on participants recalling experiences like stress, fun, and frustration in artificial settings. This memory-based method introduces secondary processing. It filters out the context and nuances of emotions. In contrast, the Reddit corpus preserves authentic expressions during emotional peaks or breakdowns. This enables identification of processes such as the buildup of cognitive stress (mental overload or strain), the triggering of frustration, and the amplification of negative experiences through social conflict. These processes are seen in language patterns. DistilBERT sentiment classification and BERTopic clustering are used for more than annotation. They build a repeatable, verifiable, and scalable framework for identifying interactive needs. We transform cognitive, emotional, social, and behavioral stress into psychological signals. These can be encoded, measured, mapped, and compared across communities. The main innovation is not simply using Reddit as a data source. Rather, we create a computable method that translates natural language into interaction needs and emotional output. This moves game experience research from descriptive results to explanatory mechanisms.